In case of \gls{acr:ssha}, the input data consist of a single earthquake
rupture model and one or more ground-motion models. Using the Ground Motion Field Calculator, multiple realizations of ground shaking can be computed, each realization sampling the aleatory uncertainties in the ground-motion model. The main calculator used to perform this analysis is the \emph{Ground Motion Field Calculator}, which was already introduced during the description of the event based PSHA workflow (see Section~\ref{subsec:event_based_psha} at page~\pageref{subsec:event_based_psha}).

As the scenario calculator does not need to determine the probability of occurrence of the specific rupture, but only sufficient information to parameterise the location (as a three-dimensional surface), the magnitude and the style-of-faulting of the rupture, a more simplified NRML structure is needed. A \emph{rupture model} XML can be defined in the following formats:

\begin{enumerate}
    \item \emph{Simple Fault Rupture} - in which the geometry is defined by the trace of the fault rupture, the dip and the upper and lower seismogenic depths. An example is shown below:
\begin{minted}[firstline=1,firstnumber=1,fontsize=\footnotesize,frame=single,bgcolor=lightgray]{xml}
<?xml version='1.0' encoding='utf-8'?>
<nrml xmlns:gml="http://www.opengis.net/gml"
      xmlns="http://openquake.org/xmlns/nrml/0.5">
    <simpleFaultRupture>
        <magnitude>7.0</magnitude>
        <rake>90.0</rake>
        <hypocenter lat="0.0" lon="0.0" depth="10.0"/>
        <simpleFaultGeometry>
            <gml:LineString>
                <gml:posList>
                    0.0 -0.3
                    0.0  0.3
                </gml:posList>
            </gml:LineString>
            <dip>90.0</dip>
            <upperSeismoDepth>2.0</upperSeismoDepth>
            <lowerSeismoDepth>20.0</lowerSeismoDepth>
        </simpleFaultGeometry>
    </simpleFaultRupture>
</nrml>
\end{minted}
\\
    \item \emph{Planar \& Multi-Planar Rupture} - in which the geometry is defined as a collection of one or more rectangular planes, each defined by four corners.

    \begin{minted}[firstline=1,firstnumber=1,fontsize=\footnotesize,frame=single,bgcolor=lightgray]{xml}
<?xml version='1.0' encoding='utf-8'?>
<nrml xmlns:gml="http://www.opengis.net/gml"
      xmlns="http://openquake.org/xmlns/nrml/0.5">
    <multiPlanesRupture>
        <magnitude>8.0</magnitude>
        <rake>90.0</rake>
        <hypocenter lat="-1.4" lon="1.1" depth="10.0"/>
            <planarSurface strike="90.0" dip="45.0">
                <topLeft lon="-0.8" lat="-2.3" depth="0.0" />
                <topRight lon="-0.4" lat="-2.3" depth="0.0" />
                <bottomLeft lon="-0.8" lat="-2.3890" depth="10.0" />
                <bottomRight lon="-0.4" lat="-2.3890" depth="10.0" />
            </planarSurface>
            <planarSurface strike="30.94744" dip="30.0">
                <topLeft lon="-0.42" lat="-2.3" depth="0.0" />
                <topRight lon="-0.29967" lat="-2.09945" depth="0.0" />
                <bottomLeft lon="-0.28629" lat="-2.38009" depth="10.0" />
                <bottomRight lon="-0.16598" lat="-2.17955" depth="10.0" />
            </planarSurface>
    </multiPlanesRupture>
</nrml> 
\end{minted}
\\
    \item \emph{Complex Fault Rupture} - in which the geometry is defined by the upper, lower and (if applicable) intermediate edges of the fault rupture.

\begin{minted}[firstline=1,firstnumber=1,fontsize=\footnotesize,frame=single,bgcolor=lightgray]{xml}
    <?xml version='1.0' encoding='utf-8'?>
<nrml xmlns:gml="http://www.opengis.net/gml"
      xmlns="http://openquake.org/xmlns/nrml/0.5">
    <complexFaultRupture>
        <magnitude>8.0</magnitude>
        <rake>90.0</rake>
        <hypocenter lat="-1.4" lon="1.1" depth="10.0"/>
        <complexFaultGeometry>
            <faultTopEdge>
                <gml:LineString>
                    <gml:posList>
                        0.6 -1.5 2.0
                        1.0 -1.3 5.0
                        1.5 -1.0 8.0
                    </gml:posList>
                </gml:LineString>
            </faultTopEdge>
            <intermediateEdge>
                <gml:LineString>
                    <gml:posList>
                        0.65 -1.55 4.0
                        1.1  -1.4  10.0
                        1.5  -1.2  20.0
                    </gml:posList>
                </gml:LineString>
            </intermediateEdge>
            <faultBottomEdge>
                <gml:LineString>
                    <gml:posList>
                        0.65 -1.7 8.0
                        1.1  -1.6 15.0
                        1.5  -1.7 35.0
                    </gml:posList>
                </gml:LineString>
            </faultBottomEdge>
        </complexFaultGeometry>
    </complexFaultRupture>
</nrml>
\end{minted}
\end{enumerate}
