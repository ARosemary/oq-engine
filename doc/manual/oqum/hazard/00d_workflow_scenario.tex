In case of \gls{acr:ssha}, the input data consist of a single earthquake
rupture model and one or more ground-motion models. Using the Ground Motion
Field Calculator, multiple realizations of ground shaking can be computed,
each realization sampling the aleatory uncertainties in the ground-motion
model. The main calculator used to perform this analysis is the \emph{Ground
Motion Field Calculator}, which was already introduced during the description
of the event based PSHA workflow (see Section~\ref{subsec:event_based_psha} at
page~\pageref{subsec:event_based_psha}).

As the scenario calculator does not need to determine the probability of
occurrence of the specific rupture, but only sufficient information to
parameterise the location (as a three-dimensional surface), the magnitude and
the style-of-faulting of the rupture, a more simplified NRML structure is
sufficient compared to the source model structures described previously in
Section~\ref{sec:source_typologies}.
A \emph{rupture model} XML can be defined in the following formats:

\begin{enumerate}

    \item \emph{Simple Fault Rupture} - in which the geometry is defined by the
    trace of the fault rupture, the dip and the upper and lower seismogenic
    depths. An example is shown below in Listing~\ref{lst:input_rupture_simple}.

\begin{listing}[htbp]
  \inputminted[firstline=1,firstnumber=1,fontsize=\footnotesize,frame=single,linenos,bgcolor=lightgray]{xml}{oqum/hazard/verbatim/input_rupture_simple_fault.xml}
  \caption{An example simple fault rupture input file}
  \label{lst:input_rupture_simple}
\end{listing}

    \item \emph{Planar \& Multi-Planar Rupture} - in which the geometry is
    defined as a collection of one or more rectangular planes, each defined
    by four corners. An example of a multi-planar rupture is shown below
    in Listing~\ref{lst:input_rupture_multi_planes}.

\begin{listing}[htbp]
  \inputminted[firstline=1,firstnumber=1,fontsize=\footnotesize,frame=single,linenos,bgcolor=lightgray]{xml}{oqum/hazard/verbatim/input_rupture_multi_planes.xml}
  \caption{An example multi-planar rupture input file}
  \label{lst:input_rupture_multi_planes}
\end{listing}

    \item \emph{Complex Fault Rupture} - in which the geometry is defined by
    the upper, lower and (if applicable) intermediate edges of the fault
    rupture. An example of a complex fault rupture is shown below in
    Listing~\ref{lst:input_rupture_complex}.

\begin{listing}[htbp]
  \inputminted[firstline=1,firstnumber=1,fontsize=\footnotesize,frame=single,linenos,bgcolor=lightgray]{xml}{oqum/hazard/verbatim/input_rupture_complex.xml}
  \caption{An example complex fault rupture input file}
  \label{lst:input_rupture_complex}
\end{listing}

\end{enumerate}
