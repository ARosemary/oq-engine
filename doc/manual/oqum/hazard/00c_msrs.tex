We provide below a list of the magnitude-area scaling relationships
implemented in the \gls{acr:hazlib}:

\subsection{Relationships for shallow earthquakes in active tectonic regions}

\begin{itemize}

    \item \cite{wells1994} - One of the most well known magnitude scaling
	relationships, based on a global database of historical earthquake
	ruptures. The implemented relationship is the one linking magnitude to
	rupture area, and is called with the keyword \verb=WC1994=

\end{itemize}


\subsection{Magnitude-scaling relationships for subduction earthquakes}
\begin{itemize}
    \item \cite{Strasser2010} - Defines several magnitude scaling relationships for interface and in-slab earthquakes. Only the magnitude to rupture-area scaling relationships are implemented here, and are called with the keywords \verb=StrasserInterface= and \verb=StrasserIntraslab= respectively.
    \item \cite{Thingbaijam2017} - Define  magnitude scaling relationships for interface. Only the magnitude to rupture-area scaling relationships are implemented here, and are called with the keywords \verb=ThingbaijamInterface=.
\end{itemize}

\subsection{Magnitude-scaling relationships stable continental regions}
\begin{itemize}
    \item \cite{ceus2011} - Defines a single magnitude to rupture-area scaling relationship for use in the central and eastern United States: $Area = 10.0^{M_W - 4.336}$. It is called with the keyword \verb=CEUS2011=
\end{itemize}

\subsection{Miscellaneous Magnitude-Scaling Relationships}
\begin{itemize}
    \item \verb=PeerMSR= defines a simple magnitude scaling relation used as part of the Pacific Earthquake Engineering Research Center verification of probabilistic seismic hazard analysis programs: $Area = 10.0 ^{M_W - 4.0}$.
    \item \verb=PointMSR= approximates a `point' source by returning an infinitesimally small area for all magnitudes. Should only be used for distributed seismicity sources and not for fault sources. 
\end{itemize}

%
%\subsection{Ground motion prediction equations for volcanic areas}
%\begin{itemize}
%    \item
%\end{itemize}
