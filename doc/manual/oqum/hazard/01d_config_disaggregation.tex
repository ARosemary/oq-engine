In this section we describe the structure of the configuration file to be used
to complete a seismic hazard disaggregation. Since only a few parts of the
standard configuration file need to be changed we can use the description
given in Section~\ref{subsec:config_classical_psha} at
page~\pageref{subsec:config_classical_psha} as a reference and we emphasize
herein major differences.


\begin{minted}[firstline=1,linenos=true,firstnumber=1,fontsize=\footnotesize,frame=single,bgcolor=lightgray]{ini}
[general]
description = A demo .ini file for PSHA disaggregation
calculation_mode = disaggregation
random_seed = 1024
\end{minted}

The calculation mode parameter in this case is set as
\texttt{disaggregation}.



\textbf{Geometry of the area (or the sites) where hazard is computed}

\begin{minted}[firstline=1,linenos=true,firstnumber=5,fontsize=\footnotesize,frame=single,bgcolor=lightgray]{ini}
[geometry]
sites = 11.0 44.5
\end{minted}

In the section it is necessary to specify the geographic coordinates of
the site (or sites) where the disaggregation will be performed.



\textbf{Disaggregation parameters}

\begin{minted}[firstline=1,linenos=true,firstnumber=7,fontsize=\footnotesize,frame=single,bgcolor=lightgray]{ini}
[disaggregation]
poes_disagg = 0.02, 0.1
mag_bin_width = 1.0
distance_bin_width = 25.0
coordinate_bin_width = 1.5
num_epsilon_bins = 3
\end{minted}

With the disaggregation settings shown above we'll disaggregate the intensity
measure levels with 10\% and 2\% probability of exceedance using the
\texttt{in\-ves\-ti\-gation\_time} and the intensity measure types  defined in
the ``Calculation configuration'' section of the OpenQuake configuration file
(see page~\pageref{sec:calculation_configuration}).

The parameters \texttt{mag\_bin\_width},  \texttt{distance\_bin\_width},
\texttt{coordinate\_bin\_width} control the level of discretization of the
disaggregation matrix computed. \texttt{num\_epsilon\_bins} indicates the
number of bins used to represent the contributions provided by different
values of epsilon.

If the user is interested in a specific type of disaggregation, we suggest to
use a very coarse gridding for the parameters that are  not necessary. For
example, if the user is interested in a magnitude-distance  disaggregation, we
suggest the use of very large value for the
\texttt{coordinate\_\-bin\_\-width} and to set  \texttt{num\_epsilon\_bins}
equal to 1.
