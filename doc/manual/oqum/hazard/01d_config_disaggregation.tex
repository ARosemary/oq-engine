In this section we describe the structure of the configuration file to be used
to complete a seismic hazard disaggregation. Since only a few parts of the
standard configuration file need to be changed we can use the description
given in Section~\ref{subsec:config_classical_psha} at
page~\pageref{subsec:config_classical_psha} as a reference and we emphasize
herein major differences.

\begin{minted}[firstline=1,linenos=true,firstnumber=1,fontsize=\footnotesize,frame=single,bgcolor=lightgray]{ini}
[general]
description = A demo .ini file for PSHA disaggregation
calculation_mode = disaggregation
random_seed = 1024
\end{minted}

The calculation mode parameter in this case is set as
\texttt{disaggregation}.

\textbf{Geometry of the area (or the sites) where hazard is computed}

\begin{minted}[firstline=1,linenos=true,firstnumber=5,fontsize=\footnotesize,frame=single,bgcolor=lightgray]{ini}
[geometry]
sites = 11.0 44.5
\end{minted}

In the section it is necessary to specify the geographic coordinates of
the site(s) where the disaggregation will be performed. The coordinates
of multiple site should be separated with a comma.

\textbf{Disaggregation parameters}

The disaggregation parameters need to be added to the the standard
configuration file. They are shown in the following example and a description
of each parameter is provided below.

\begin{minted}[firstline=1,linenos=true,firstnumber=7,fontsize=\footnotesize,frame=single,bgcolor=lightgray]{ini}
[disaggregation]
poes_disagg = 0.02, 0.1
mag_bin_width = 1.0
distance_bin_width = 25.0
coordinate_bin_width = 1.5
num_epsilon_bins = 3
disagg_outputs = Mag_Dist_Eps Mag_Lon_Lat
\end{minted}

\begin{itemize}

    \item \Verb+poes_disagg+: disaggregation is performed for the intensity
    measure levels corresponding to the probability of exceedance value(s) provided
    here. The computations use the \texttt{investigation\_time} and the
    \texttt{intensity\_measure\_types\_and\_levels} defined in the
    ``Calculation configuration'' section   (see page~\pageref{sec:calculation_configuration}).
    For the \texttt{poes\_disagg} the intensity measure level(s) for the disaggregation are
    inferred by performing a classical calculation and by inverting the hazard curves.

    \item \Verb+iml_disagg+: the intensity measure level(s) to be disaggregated can be directly defined
    by specifying \texttt{iml\_disagg}. Note that a
    disaggregation computation requires either \texttt{poes\_disagg} or
    \texttt{iml\_disagg} to be defined, but both cannot be defined at the same time.

    \item \Verb+mag_bin_width+: mandatory; specifies the width of every magnitude
     histogram bin of the disaggregation matrix computed

    \item \Verb+distance_bin_width+: specifies the width of every distance
    histogram bin of the disaggregation matrix computed (km)

    \item \Verb+coordinate_bin_width+: specifies the width of every longitude-latitude
    histogram bin of the disaggregation matrix computed (decimal degrees)

    \item \Verb+num_epsilon_bins+: mandatory; specifies the number of epsilon
    histogram bins of the disaggregation matrix. The width of the epsilon bins
    depends on the \texttt{truncation\_level} defined in the
    ``Calculation configuration'' section (page~\pageref{sec:calculation_configuration})

    \item \Verb+disagg_outputs+: optional; specifies the type(s) of disaggregation
    to be computed. The options are: \texttt{Mag}, \texttt{Dist}, \texttt{Lon\_Lat},
    \texttt{Lon\_Lat\_TRT}, \texttt{Mag\_Dist}, \texttt{Mag\_Dist\_Eps},
    \texttt{Mag\_Lon\_Lat}, \texttt{TRT}. If none are specified, then all are
    computed. More details of the disaggregation output are given in the
    ``Outputs from Hazard Disaggregation'' section,
    see page~\pageref{subsec:output_hazard_disaggregation})

    \item \Verb+disagg_by_src+: optional; if specified and set to true, disaggregation
    by source is computed. This option currently only works if the logic tree is trivial
    (i.e. there is only one realization).

\end{itemize}

As mentioned above, the user also has the option to perform disaggregation by
directly specifying the intensity measure level to be disaggregated, rather than
specifying the probability of exceedance. An example is shown below:

\begin{minted}[firstline=1,linenos=true,firstnumber=7,fontsize=\footnotesize,frame=single,bgcolor=lightgray]{ini}
[disaggregation]
iml_disagg = {'PGA': 0.1}
\end{minted}

If \texttt{iml\_disagg} is specified, the user should not include
 \texttt{intensity\_measure\_types\_and\_levels} in the
``Calculation configuration'' section (see page~\pageref{sec:calculation_configuration})
since it is explicitly given here.

\textbf{Statistical Disaggregation Outputs}

 When there is more than 1 logic tree realization, the disaggregation is
 computed for each realization. Additionally, if \texttt{mean\_hazard\_curves}
 is set to true, or \texttt{quantile\_hazard\_curves} are specified (see
 ``Calculation configuration'' section, page~\pageref{sec:calculation_configuration})
 then disaggregation will also be performed for the mean hazard and/or the
 specified quantiles.