Input data for the classical \gls{acr:psha} consist of a PSHA input model
provided together with calculation settings.

The main calculators used to perform this analysis are the following:

\begin{enumerate}

	\item \emph{Logic Tree Processor}

	The Logic Tree Processor (LTP) takes as an input the \gls{acr:psha} Input
	Model and creates a Seismic Source Model. The LTP uses the information in
	the Initial Seismic Source Models and the Seismic Source Logic Tree to
	create a Seismic Source Input Model (i.e. a model describing geometry and
	activity rates of each source without any epistemic uncertainty).

	Following a procedure similar to the one just described the Logic Tree
	Processor creates a Ground Motion model (i.e. a data structure that
	associates to each tectonic region considered in the calculation a
	\gls{acr:gmpe}).

	\item \emph{Earthquake Rupture Forecast Calculator}

	The produced Seismic Source Input Model becomes an input information for
	the Earthquake Rupture Forecast (ERF) calculator which creates a list
	earthquake ruptures admitted by the source model, each one characterized
	by a probability of occurrence over a specified time span.

	\item \emph{Classical PSHA Calculator}

	The classical PSHA calculator uses the ERF and the Ground Motion model to
	compute hazard curves on each site specified in the calculation settings.

\end{enumerate}