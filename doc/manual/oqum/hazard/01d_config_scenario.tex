In order to run this calculator, the parameter \Verb+calculation_mode+ needs
to be set to \Verb+scenario+. 

The basic job configuration file required for running a scenario hazard
calculation is shown in Listing~\ref{lst:config_scenario_hazard}.

\begin{listing}[htbp]
  \inputminted[firstline=1,firstnumber=1,fontsize=\footnotesize,frame=single,linenos,bgcolor=lightgray,label=job.ini]{ini}{oqum/hazard/verbatim/config_scenario.ini}
  \caption{Example configuration file for a scenario hazard calculation (\href{https://raw.githubusercontent.com/gem/oq-engine/master/oqum/hazard/verbatim/config_scenario.ini}{Download example})}
  \label{lst:config_scenario_hazard}
\end{listing}

Most of the job configuration parameters required for running a scenario
hazard calculation seen in the example in
Listing~\ref{lst:config_scenario_hazard} are the same as those described in
the previous sections for the classical PSHA calculator
(Section~\ref{subsec:config_classical_psha}) and the event-based PSHA
calculator (Section~\ref{subsec:config_event_based_psha}).

The set of sites at which the ground motion fields will be produced can be
specifed by using either the \Verb+sites+ or \Verb+sites_csv+ parameters, or
the \Verb+region+ and \Verb+region_grid_spacing+  parameters, similar to the
classical PSHA and event-based PSHA calculators.

The parameter unique to the scenario calculator is described below:

\begin{itemize}

  \item \Verb+number_of_ground_motion_fields+: this parameter is used to
    specify the number of Monte Carlo simulations of the ground motion
    values at the specified sites

  \item \Verb+gsim+: this parameter indicates the name of a ground motion
  prediction equation (a list of available GMPEs can be obtained using
  \texttt{oq info -g} or \texttt{oq info -{}-gsims} and these are also
  documented at: \href{http://docs.openquake.org/oq-engine/2.8/openquake.hazardlib.gsim.html}{http://docs.openquake.org/oq-engine/2.8/openquake.hazardlib.gsim.html})

\end{itemize}

Multiple ground motion prediction equations can be used for a scenario hazard
calculation by providing a GMPE logic tree file (described previously in 
Section~\ref{subsec:gmlt}) using the parameter \Verb+gsim_logic_tree_file+.
In this case, the \glsdesc{acr:oqe} generates ground motion fields
for all GMPEs specified in the logic tree file. The branch weights in the logic
tree file are ignored in a scenario analysis and only the individual branch
results are computed. Mean or quantile ground motion fields will not be 
generated.

The ground motion fields will be computed at each of the sites and for each of
the intensity measure types specified in the job configuration file. The
above calculation can be run using the command line:

\begin{minted}[fontsize=\footnotesize,frame=single,bgcolor=lightgray]{shell-session}
user@ubuntu:~\$ oq engine --run job.ini
\end{minted}

After the calculation is completed, a message similar to the following will be
displayed:

\begin{minted}[fontsize=\footnotesize,frame=single,bgcolor=lightgray]{shell-session}
Calculation 260 completed in 3 seconds. Results:
  id | name
 569 | gmf_data
\end{minted}