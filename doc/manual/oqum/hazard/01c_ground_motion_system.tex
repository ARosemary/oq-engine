The Ground Motion System defines the models and the possible epistemic
uncertainties related to ground motion modelling to be incorporated into the
calculation.

\subsection{The Ground Motion Logic Tree}
\index{Input!Ground motion logic tree}
\label{subsec:gmlt}

The structure of the \gls{groundmotionlogictree} consists of a list of ground
motion prediction equations for each tectonic region used to characterise the
sources in the PSHA input model.

The example below shows a simple \gls{groundmotionlogictree}. This logic tree
assumes that all the sources in the PSHA input model belong to ``Active
Shallow Crust'' and uses for calculation the \citet{chiou2008}
\gls{acr:gmpe}.

\begin{minted}[firstline=1,firstnumber=1,fontsize=\footnotesize,frame=single,bgcolor=lightgray]{xml}
<?xml version="1.0" encoding="UTF-8"?>
<nrml xmlns:gml="http://www.opengis.net/gml"
      xmlns="http://openquake.org/xmlns/nrml/0.5">
    <logicTree logicTreeID="lt1">
        <logicTreeBranchingLevel branchingLevelID="bl1">
            <logicTreeBranchSet uncertaintyType="gmpeModel"
                    branchSetID="bs1"
                    applyToTectonicRegionType="Active Shallow Crust">

                <logicTreeBranch branchID="b1">
                    <uncertaintyModel>
                    ChiouYoungs2008
                    </uncertaintyModel>
                    <uncertaintyWeight>1.0</uncertaintyWeight>
                </logicTreeBranch>

            </logicTreeBranchSet>
        </logicTreeBranchingLevel>
    </logicTree>
</nrml>
\end{minted}
%\begin{Verbatim}[frame=single, commandchars=\\\{\}, fontsize=\small,
    firstnumber=1, numbers=left, numbersep=2pt]
<?xml version="1.0" encoding="UTF-8"?>
<nrml xmlns:gml="http://www.opengis.net/gml"
      xmlns="http://openquake.org/xmlns/nrml/0.5">
    <logicTree logicTreeID="lt1">
        <logicTreeBranchingLevel branchingLevelID="bl1">
            <logicTreeBranchSet uncertaintyType="gmpeModel"
                    branchSetID="bs1"
                    applyToTectonicRegionType="Active Shallow Crust">

                <logicTreeBranch branchID="b1">
                    <uncertaintyModel>
                    ChiouYoungs2008
                    </uncertaintyModel>
                    <uncertaintyWeight>1.0</uncertaintyWeight>
                </logicTreeBranch>

            </logicTreeBranchSet>
        </logicTreeBranchingLevel>
    </logicTree>
</nrml>
\end{Verbatim}
