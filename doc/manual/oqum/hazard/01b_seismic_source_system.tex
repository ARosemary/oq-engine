The Seismic Source System contains the model (or the models) describing
position, geometry and activity of seismic sources of engineering importance
for a set of sites as well as the possible epistemic uncertainties to be
incorporated into the calculation of seismic hazard.



\subsection{The Seismic Source Logic Tree}

The structure of the Seismic Source Logic Tree consists of at least one
\gls{branchinglevel}. This branching level is the one used to define the
\gls{initialseismicsourceinputmodel} (or a number of initial seismic source
models, see Figure~\ref{fig:psha_input}).

The example provided below shows the simplest Seismic Source Logic Tree
structure that can be defined in a \gls{pshainputmodel} for \gls{acr:oqe}.
It's a logic tree with just one branching level containing one \gls{branchset}
with one branch used to define the initial seismic source model (its weight
will be equal to one).

\begin{listing}[htbp]
  \inputminted[firstline=1,firstnumber=1,fontsize=\footnotesize,frame=single,linenos,bgcolor=lightgray]{xml}{oqum/hazard/verbatim/input_sslt.xml}
  \caption{Example seismic source model logic tree input file}
  \label{lst:input_sslt}
\end{listing}

The optional branching levels will contain rules that modify parameters of the
sources in the initial seismic source model.

For example, if the epistemic uncertainties to be considered are source
geometry and maximum magnitude, the modeller can create a logic tree structure
with three initial seismic source models (each one exploring a different
definition of the geometry of sources) and one branching level accounting for
the epistemic uncertainty on the maximum magnitude.

Below we provide an example of such logic tree structure. Note that the
uncertainty on the maximum magnitude is specified in terms of relative
increments with respect to the initial maximum magnitude defined for each
source in the initial seismic source models.

\inputminted[firstline=1,firstnumber=1,fontsize=\footnotesize,frame=single,linenos,bgcolor=lightgray]{xml}{oqum/hazard/verbatim/input_sslt_simple_lt.xml}
\captionof{listing}{Example source model logic tree structure\label{lst:example_source_model_logic_tree}}

Starting from \glsdesc{acr:oqe24}, it is also possible to split a source model
into several files and read them as if they were a single file. The file names
for the different files comprising a source model should be provided in the
source model logic tree file. For instance, a source model could be split by
tectonic region using the following syntax in the source model logic tree:

\begin{minted}[firstline=1,firstnumber=1,fontsize=\footnotesize,frame=single,bgcolor=lightgray]{xml}
<?xml version="1.0" encoding="UTF-8"?>
<nrml xmlns:gml="http://www.opengis.net/gml"
      xmlns="http://openquake.org/xmlns/nrml/0.5">
    <logicTree logicTreeID="lt1">
        <logicTreeBranchingLevel branchingLevelID="bl1">
            <logicTreeBranchSet uncertaintyType="sourceModel"
                                branchSetID="bs1">
                <logicTreeBranch branchID="b1">
                    <uncertaintyModel>
		        active_shallow_sources.xml
		        stable_shallow_sources.xml
                    </uncertaintyModel>
                    <uncertaintyWeight>1.0</uncertaintyWeight>
                </logicTreeBranch>
            </logicTreeBranchSet>
        </logicTreeBranchingLevel>
    </logicTree>
</nrml>
\end{minted}


\subsection{The Seismic Source Model}
\index{Input!Configuration file}

The structure of the xml file representing the seismic source model
corresponds to a list of sources, each one modelled using one out of the five
typologies currently supported. Below we provide a schematic example of a
seismic source model:

\begin{listing}[htbp]
  \inputminted[firstline=1,firstnumber=1,fontsize=\footnotesize,frame=single,linenos,bgcolor=lightgray]{xml}{oqum/hazard/verbatim/input_sslt.xml}
  \caption{Example seismic source model input file}
  \label{lst:input_ssm}
\end{listing}