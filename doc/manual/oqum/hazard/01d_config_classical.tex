In the following we describe the overall structure and the most typical
parameters of a configuration file to be used for the computation of a
seismic hazard map using a classical PSHA methodology.


\textbf{Calculation type and model info}

\begin{minted}[firstline=1,linenos=true,firstnumber=1,fontsize=\footnotesize,frame=single,bgcolor=lightgray]{ini}
[general]
description = A demo OpenQuake-engine .ini file for classical PSHA
calculation_mode = classical
random_seed = 1024
\end{minted}

In this section the user specifies the following parameters:

\begin{itemize}

    \item \texttt{description}: a parameter that can be used to designate
    the model

    \item \texttt{calculation\_mode}: it is used to set the kind of
    calculation. In this case it corresponds to \texttt{classical}.
    Alternative options for the calculation\_mode are described later in this
    manual.

    \item \texttt{random\_seed}: is used to control the random generator
    so that when Monte Carlo procedures are used calculations are
    replicable (if the same \texttt{random\_seed} is used you get exactly
    the same results).

\end{itemize}

\textbf{Geometry of the area (or the sites) where hazard is computed}

This section is used to specify where the hazard will be computed. Two
options are available:

The first option is to define a polygon (usually a rectangle) and a distance
(in km) to be used to discretize the  polygon area. The polygon is defined by
a list of longitude-latitude tuples.

An example is provided below:

\begin{minted}[firstline=1,linenos=true,firstnumber=5,fontsize=\footnotesize,frame=single,bgcolor=lightgray]{ini}
[geometry]
region = 10.0 43.0, 12.0 43.0, 12.0 46.0, 10.0 46.0
region_grid_spacing = 10.0
\end{minted}

The second option allows the definition of a number of sites where the hazard
will be computed. Each site is specified in terms of a longitude, latitude tuple.
Optionally, if the user wants to consider the elevation of the sites, a value of
depth [km] can also be specified, where positive values indicate below sea level,
and negative values indicate above sea level (i.e. the topographic surface). If
no value of depth is given for a site, it is assumed to be zero. An example is
provided below:

\begin{minted}[firstline=1,linenos=true,firstnumber=8,fontsize=\footnotesize,frame=single,bgcolor=lightgray]{ini}
[geometry]
sites = 10.0 43.0, 12.0 43.0, 12.0 46.0, 10.0 46.0
\end{minted}

If the list of sites is too long the user can specify the name of a csv file
as shown below:

\begin{minted}[firstline=1,linenos=true,firstnumber=10,fontsize=\footnotesize,frame=single,bgcolor=lightgray]{ini}
[geometry]
sites_csv = <name_of_the_csv_file>
\end{minted}

The format of the csv file containing the list of sites is a sequence of
points (one per row) specified in terms of the longitude, latitude tuple. Depth
values are again optional. An example is provided below:

\begin{minted}[firstline=1,linenos=false,firstnumber=10,fontsize=\footnotesize,frame=single,bgcolor=lightgray]{text}
179.0,90.0
178.0,89.0
177.0,88.0
\end{minted}

\textbf{Logic tree sampling}

The \gls{acr:oqe} provides two options for processing the whole logic tree
structure. The first option uses Montecarlo sampling; the user in this case
specifies a number of realizations.

In the second option all the possible realizations are created. Below we
provide an example for the latter option. In this case we set the
\texttt{number\-\_of\-\_logic\_tree\_samples} to 0. \gls{acr:oqe} will perform
a complete enumeration of all  the possible paths from the roots to the leaves
of the logic tree  structure.

\begin{minted}[firstline=1,linenos=true,firstnumber=12,fontsize=\footnotesize,frame=single,bgcolor=lightgray]{ini}
[logic_tree]
number_of_logic_tree_samples = 0
\end{minted}

If the seismic source logic tree and the ground motion logic tree do not
contain epistemic uncertainties the engine will create a single PSHA input.

\textbf{Generation of the earthquake rupture forecast}

\begin{minted}[firstline=1,linenos=true,firstnumber=14,fontsize=\footnotesize,frame=single,bgcolor=lightgray]{ini}
[erf]
rupture_mesh_spacing = 5
width_of_mfd_bin = 0.1
area_source_discretization = 10
\end{minted}

This section of the configuration file is used to specify the level of
discretization of the mesh representing faults, the grid used to delineate the
area sources and, the magnitude-frequency distribution. Note that the smaller
is the mesh spacing (or the bin width) the larger are (1) the precision in the
calculation and (2) the computation demand.

In cases where the source model may contain a mixture of simple and complex
ruptures it is possible to define a different rupture mesh spacing for complex
faults only. This may be helpful in models that permit floating ruptures over
large subduction sources, in which the nearest source to site distances may be
larger than 20 - 30 km, and for which a small mesh spacing would produce a
very large number of ruptures. The spacing for complex faults only can be
configured by the line:

\begin{minted}[firstline=1,linenos=true,firstnumber=18,fontsize=\footnotesize,frame=single,bgcolor=lightgray]{ini}
complex_fault_mesh_spacing = 10
\end{minted}

\textbf{Parameters describing site conditions}

\begin{minted}[firstline=1,linenos=true,firstnumber=18,fontsize=\footnotesize,frame=single,bgcolor=lightgray]{ini}
[site_params]
reference_vs30_type = measured
reference_vs30_value = 760.0
reference_depth_to_2pt5km_per_sec = 5.0
reference_depth_to_1pt0km_per_sec = 100.0
\end{minted}

In this section the user specifies local soil conditions. The simplest
solution is to define uniform site conditions (i.e. all the sites have  the
same characteristics).

Alternatively it is possible to define spatially variable soil properties in
a separate file; the engine will then assign to each investigation location
the values of the closest point used to specify site conditions.

\begin{minted}[firstline=1,linenos=true,firstnumber=23,fontsize=\footnotesize,frame=single,bgcolor=lightgray]{ini}
[site_params]
site_model_file = site_model.xml
\end{minted}

The file containing the site model has the following structure:

\begin{listing}[htbp]
  \inputminted[firstline=1,firstnumber=1,fontsize=\footnotesize,frame=single,linenos,bgcolor=lightgray]{xml}{oqum/hazard/verbatim/input_site_model.xml}
  \caption{Example site model input file}
  \label{lst:input_site_model}
\end{listing}

If the closest available site with soil conditions is at a distance greater than 5~km from the investigation location, a warning is generated.

\textbf{Calculation configuration}
\phantomsection
\label{sec:calculation_configuration}

\begin{minted}[firstline=1,linenos=true,firstnumber=25,fontsize=\footnotesize,frame=single,bgcolor=lightgray]{ini}
[calculation]
source_model_logic_tree_file = source_model_logic_tree.xml
gsim_logic_tree_file = gmpe_logic_tree.xml
investigation_time = 50.0
intensity_measure_types_and_levels = {"PGA": [0.005, ..., 2.13]}
truncation_level = 3
maximum_distance = 200.0
\end{minted}

This section of the \gls{acr:oqe} configuration file specifies the parameters
that are relevant for the calculation of hazard. These include the names of
the two files containing the Seismic Source System and the Ground Motion
System, the duration of the time window used to compute the  hazard, the
ground motion intensity measure types and levels for  which the probability of
exceedence will be computed, the level of truncation of the Gaussian
distribution of the logarithm of ground motion used in the calculation of
hazard and the maximum integration distance (i.e. the distance within which
sources will contribute to the computation of the hazard).

The maximum distance refers to the largest distance between a rupture and the
target calculation sites in order for the rupture to be considered in the PSHA
calculation. This can be input directly in terms of kilometres (as above).
There may be cases, however, in which it may be appropriate to have a
different maximum source to site distance depending on the tectonic region
type. This may be used, for example, to eliminate the impact of small, very
far-field sources in regions of high attenuation (in which case maximum distance
is reduced), or conversely it may be raised to allow certain source types to
contribute to the hazard at greater distances (such as in the case when the
region has lower attenuation). An example configuration for a maximum distance
in Active Shallow Crust of 150 km, and in Stable Continental Crust of 200 km,
is shown below:

\begin{minted}[firstline=1,linenos=true,firstnumber=31,fontsize=\footnotesize,frame=single,bgcolor=lightgray]{ini}
maximum_distance = {'Active Shallow Crust': 150.0,
                    'Stable Continental Crust': 200.0}
\end{minted}

An even more advanced approach is to use a maximum distance depending on the
magnitude (magnitude-distance filter): in that case the user must specify
the maximum distance per a discrete set of magnitudes. Here is an
example:

\begin{minted}[fontsize=\footnotesize,frame=single,bgcolor=lightgray]{ini}
maximum_distance = {'Active Shallow Crust': [(8, 250), (7, 150), (5, 50)],
                    'Stable Continental Crust': [(8, 300), (7, 200), (5, 100)]}
\end{minted}

You should read this example as follows: for Active Shallow Crust

\begin{itemize}
\item keep sites within 250 km from the rupture if the magnitude is <= 8
\item keep sites within 150 km from the rupture if the magnitude is <= 7
\item keep sites within 50 km from the rupture if the magnitude is <= 5
\end{itemize}

while for Stable Continental Crust

\begin{itemize}
\item keep sites within 300 km from the rupture if the magnitude is <= 8
\item keep sites within 200 km from the rupture if the magnitude is <= 7
\item keep sites within 100 km from the rupture if the magnitude is <= 5
\end{itemize}

It is also possible to define something like the following:

\begin{minted}[fontsize=\footnotesize,frame=single,bgcolor=lightgray]{ini}
maximum_distance = [(8, 250), (7, 150), (5, 50)]
\end{minted}

In this case the same magnitude-distance filter is used for all tectonic
region types.

If the magnitude is above the maximum magnitude of the filter
(in this example 8) we keep the sites within 2000 km of the ruptures, i.e.
effectively we do not filter.

Notice that the filtering has a big impact on the performance, by
reducing the maximum distance for small magnitudes one can easily
speed up the calculations by 2-3 times or more, without losing much
precision.

\textbf{Output}

\begin{minted}[firstline=1,linenos=true,firstnumber=31,fontsize=\footnotesize,frame=single,bgcolor=lightgray]{ini}
[output]
export_dir = outputs/
# given the specified `intensity_measure_types_and_levels`
mean_hazard_curves = true
quantile_hazard_curves = 0.1 0.5 0.9
uniform_hazard_spectra = false
poes = 0.1
\end{minted}

The final section of the configuration file is the one that contains the
parameters controlling the types of output to be produced. Providing an export
directory will tell OpenQuake where to place the output files when the
\texttt{-{}-exports} flag is used when running the program. Setting
\verb=mean_hazard_curves= to true will result in a specific output containing
the mean curves of the logic tree, likewise \verb=quantile_hazard_curves= will
produce separate files containing the quantile hazard curves at the quantiles
listed (0.1, 0.5 and 0.9 in the example above, leave blank or omit if no
quantiles are required). Setting \verb=uniform_hazard_spectra= to true will
output the uniform hazard spectra at the same probabilities of exceedence
(poes) as those specified by the later option \verb=poes=. The probabilities
specified here correspond to the set investigation time. Specifying poes
will output hazard maps. For more information about the outputs of the
calculation, see the section:
``Description of hazard output'' (page~\pageref{sec:hazard_outputs}).
