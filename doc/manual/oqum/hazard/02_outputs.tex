In this Chapter we provide a desciption of the main commands available for
running hazard with the \gls{acr:oqe} and the file formats used to represent
the results of the analyses.

A general introduction on the use of the \glsdesc{acr:oqe} is provided in
Chapter~\ref{chap:intro} at page~\pageref{chap:intro}. The
reader is invited to consult this part before diving into the following
sections.


% -----------------------------------------------------------------------------
\section{Running OpenQuake-engine for hazard calculations}
\label{sec:running_hazard_calculations}
\index{Running OpenQuake!hazard}

The execution of a hazard analysis using the OpenQuake-engine is
straightforward. Below we provide an example of the simplest command that can be
used to launch a hazard calculation. It consists in the invocation of \texttt
{oq engine} together with the \texttt{-{}-run} option,
and the name of a configuration file (in the example below it
corresponds to \texttt{job.ini}):

\begin{minted}[firstline=1,linenos=false,firstnumber=1,fontsize=\footnotesize,frame=single,bgcolor=lightgray]{bash}
user@ubuntu:$ oq engine --run job.ini
\end{minted}

The amount of information prompted during the execution of the analysis can be
controlled through the \texttt{-{}-log-level} flag as shown in the example below:

\begin{minted}[firstline=1,linenos=false,firstnumber=1,fontsize=\footnotesize,frame=single,bgcolor=lightgray]{bash}
user@ubuntu:$ oq engine --run job.ini --log-level debug
\end{minted}

In this example we ask the engine to provide an extensive amount of information
(usually not justified for a standard analysis). Alternative options are:
\texttt{debug}, \texttt{info}, \texttt{warn}, \texttt{error},
\texttt{critical}.


% -----------------------------------------------------------------------------
\section{Exporting results from a hazard calculation}
\label{sec:exporting_hazard_results}

There are two alternative ways to get results from the OpenQuake-engine:
directly through the calculation or by exporting them from the internal
\gls{acr:oqe} database once a calculation is completed.

The first option is defined at the OpenQuake-engine invocation through the flag \texttt{--exports xml}, as shown in the example below:

\begin{minted}[firstline=1,linenos=false,firstnumber=1,fontsize=\footnotesize,frame=single,bgcolor=lightgray]{bash}
user@ubuntu:~$ oq engine --run job.ini --exports xml
\end{minted}

This will export the results to the \verb=results= directory specified in the \verb=job.ini= file. 

The second option allows the user to export the computed results or just a subset of them whenever they want. In order to obtain the list of results of the hazard calculations stored in the \gls{acr:oqe} database the user can utilize the \texttt{-{}-lhc} command (`list hazard calculations') to list the hazard calculations:

\begin{minted}[firstline=1,linenos=false,firstnumber=1,fontsize=\footnotesize,frame=single,bgcolor=lightgray]{bash}
user@ubuntu:~$ oq engine --lhc
\end{minted}

The execution of this command will produce a list similar to the one provided
below (the numbers in red are the calculations IDs):

\begin{Verbatim}[frame=single, commandchars=\\\{\}, fontsize=\small]
user@ubuntu:~$ oq engine --lhc
job_id | status | start_time | description
\textcolor{red}{1} | failed | 2013-03-01 09:49:34 | Classical PSHA
\textcolor{red}{2} | successful | 2013-03-01 09:49:56 | Classical PSHA
\textcolor{red}{3} | failed | 2013-03-01 10:24:04 | Classical PSHA
\textcolor{red}{4} | failed | 2013-03-01 10:28:16 | Classical PSHA
\textcolor{red}{5} | failed | 2013-03-01 10:30:04 | Classical PSHA
\textcolor{red}{6} | successful | 2013-03-01 10:31:53 | Classical PSHA
\textcolor{red}{7} | failed | 2013-03-09 08:15:14 | Classical PSHA
\textcolor{red}{8} | successful | 2013-03-09 08:18:04 | Classical PSHA
\end{Verbatim}

Subsequently the user can get the list of result stored for a specific hazard
analysis by using the \texttt{-{}-list-outputs}, or \texttt{-{}-lo}, command, as in the example below (note that the number in blue emphasizes the
result ID):

\begin{Verbatim}[frame=single, commandchars=\\\{\}, fontsize=\small]
user@ubuntu:~$ oq engine --lo <calc_id>
id | name
\textcolor{blue}{3} | hcurves
\end{Verbatim}

and finally extract an xml file for a specific hazard result:

\begin{Verbatim}[frame=single, commandchars=\\\{\}, fontsize=\small]
user@ubuntu:~$ oq engine --export-outputs <result_id> <output_folder>
\end{Verbatim}


% -----------------------------------------------------------------------------
\section{Description of hazard outputs}
\label{sec:hazard_outputs}

The results generated by the OpenQuake-engine are fundamentally of two
distinct typologies differentiated by the presence (or absence) of epistemic
uncertainty in the PSHA input model.

When epistemic uncertainty is incorporated into the calculation, the
OpenQuake-engine calculators (e.g. Classical PSHA, Event Based PSHA,
Disaggregation, UHS) produce a set of results (i.e. hazard curves, ground
motion fields, disaggregation matrices, UHS, for each logic-tree realisation)
which reflects epistemic uncertainties introduced in the PSHA input model.

For each logic tree sample, results are computed and stored. Calculation of
results statistics (mean, standard deviation, quantiles) are supported by all
the calculators, with the exception of the disaggregation calculator.

\subsection{Outputs from Classical PSHA}
\label{subsec:output_classical_psha}
By default, the classical PSHA calculator computes and stores hazard curves
for each logic tree sample considered.

When the PSHA input model doesn't contain epistemic uncertainties the results
is a set of hazard curves (one for each investigated site). The command below
illustrates how is possible to retrieve the group of hazard curves obtained
for a calculation with a given identifier \texttt{<calc\_id>} (see
Section~\ref{sec:exporting_hazard_results} for an explanation about how to
obtain the list of calculations performed with their corresponding ID):

\begin{Verbatim}[frame=single, commandchars=\\\{\}, fontsize=\small]
user@ubuntu:~$ oq engine --lo <calc_id>
id | name
\textcolor{red}{3 | Hazard Curves}
\textcolor{black}{4 | Realizations}
\end{Verbatim}

To export from the database the outputs (in this case hazard curves) contained
in one of the output identifies, one can do so with the following command:

\begin{Verbatim}[frame=single, commandchars=\\\{\}, fontsize=\small]
user@ubuntu:~$ oq engine --export-output <output_id> <output_directory>
\end{Verbatim}

Alternatively, if the user wishes to export all of the outputs associated with
a particular calculation then they can use the \texttt{-{}-export-outputs}
with the corresponding calculation key:

\begin{Verbatim}[frame=single, commandchars=\\\{\}, fontsize=\small]
user@ubuntu:~$ oq engine --export-outputs <calc_id> <output_directory>
\end{Verbatim}

The exports will produce one or more nrml files containing the seismic hazard
curves, as represented below in Listing~\ref{lst:output_hazard_curves_xml}.

\begin{listing}[htbp]
  \inputminted[firstline=1,firstnumber=1,fontsize=\footnotesize,frame=single,linenos,bgcolor=lightgray]{xml}{oqum/hazard/verbatim/output_hazard_curves.xml}
  \caption{Example hazard curves NRML output file}
  \label{lst:output_hazard_curves_xml}
\end{listing}

Notwithstanding the intuitiveness of this file, let's have a brief overview of
the information included. The overall content of this file is a list of hazard
curves, one for each investigated site, computed using a PSHA input model
representing one possible realisation obtained using the complete logic tree
structure.

The attributes of the \texttt{hazardCurves} element (see text in red) specify
the path of the logic tree used to create the seismic source model
(\texttt{source\-Model\-TreePath}) and the ground motion model
(\texttt{gsim\-Tree\-Path}) plus the intensity measure type and the
investigation time used to compute the probability of exceedance.

The \texttt{IMLs} element (in green in the example) contains the values of
shaking used by the engine to compute the probability of exceedance in the
investigation time. For each site this file contains a \texttt{hazardCurve}
element which has the coordinates (longitude and latitude in decimal degrees)
of the site and the values of the probability of exceedance for all the
intensity measure levels specified in the \texttt{IMLs} element.

If the hazard calculation is configured to produce results including seismic
hazard maps and uniform hazard spectra, then the list of outputs would display
the following:

\begin{Verbatim}[frame=single, commandchars=\\\{\}, fontsize=\small]
user@ubuntu:~$ oq engine --lo <calc_id>
id | name
\textcolor{red}{3 | Hazard Curves}
\textcolor{red}{4 | Hazard Maps}
\textcolor{black}{5 | Realizations}
\textcolor{red}{6 | Uniform Hazard Spectra}
\end{Verbatim}

Listing~\ref{lst:output_hazard_map_xml}) shows a sample of the nrml file
used to describe a hazard map, and and Listing~\ref{lst:output_uhs})
shows a sample of the nrml used to describe a uniform hazard spectrum.

\begin{listing}[htbp]
  \inputminted[firstline=1,firstnumber=1,fontsize=\footnotesize,frame=single,linenos,bgcolor=lightgray]{xml}{oqum/hazard/verbatim/output_hazard_map.xml}
  \caption{Example hazard map NRML output file}
  \label{lst:output_hazard_map_xml}
\end{listing}


\begin{listing}[htbp]
  \inputminted[firstline=1,firstnumber=1,fontsize=\footnotesize,frame=single,linenos,bgcolor=lightgray]{xml}{oqum/hazard/verbatim/output_uhs.xml}
  \caption{Example uniform hazard spectrum NRML output file}
  \label{lst:output_uhs}
\end{listing}

\subsection{Outputs from Hazard Disaggregation}
\label{subsec:output_hazard_disaggregation}
The \glsdesc{acr:oqe} output of a disaggregation analysis corresponds to the
combination of a hazard curve and a multidimensional matrix containing the
results of the disaggregation.

\begin{Verbatim}[frame=single, commandchars=\\\{\}, fontsize=\small]
user@ubuntu:~$ oq engine --lo <calc_id>
id | name
\textcolor{red}{3 | Disaggregation Outputs}
\textcolor{black}{4 | Realizations}
\end{Verbatim}
%\begin{Verbatim}[frame=single, commandchars=\\\{\}]
user@ubuntu:~$ oq engine --lo <calc_id>
id | output_type | name
19 | hazard_curve | hc-rlz-3
20 | hazard_curve | hc-rlz-3
21 | hazard_curve | hc-rlz-4
22 | hazard_curve | hc-rlz-4
23 | disagg_matrix | disagg(0.02)-rlz-3-SA(0.025)-POINT(10.1 40.1)
24 | disagg_matrix | disagg(0.1)-rlz-3-SA(0.025)-POINT(10.1 40.1)
25 | disagg_matrix | disagg(0.02)-rlz-3-PGA-POINT(10.1 40.1)
26 | disagg_matrix | disagg(0.1)-rlz-3-PGA-POINT(10.1 40.1)
27 | disagg_matrix | disagg(0.02)-rlz-4-SA(0.025)-POINT(10.1 40.1)
28 | disagg_matrix | disagg(0.1)-rlz-4-SA(0.025)-POINT(10.1 40.1)
29 | disagg_matrix | disagg(0.02)-rlz-4-PGA-POINT(10.1 40.1)
30 | disagg_matrix | disagg(0.1)-rlz-4-PGA-POINT(10.1 40.1)
\end{Verbatim}

Running \texttt{-{}-export-output} to export the disaggregation results will
produce individual files for each IMT, probability of exceedence and logic tree
realisation. In Listing~\ref{lst:output_disagg_matrix}) we show an example of
the nrml file used to represent the different disaggregation matrices (highlighted
in red) produced by\gls{acr:oqe}:

\begin{listing}[htbp]
  \begin{Verbatim}[frame=single, commandchars=\\\{\}, fontsize=\small]
<?xml version="2.0" encoding="UTF-8"?>
<nrml xmlns:gml="http://www.opengis.net/gml"
      xmlns="http://openquake.org/xmlns/nrml/0.5">
  <disaggMatrices sourceModelTreePath="b1" gsimTreePath="b1" IMT="PGA"
        investigationTime="50.0" lon="10.1" lat="40.1"
        magBinEdges="5.0, 6.0, 7.0, 8.0"
        distBinEdges="0.0, 25.0, 50.0, 75.0, 100.0"
        lonBinEdges="9.0, 10.5, 12.0"
        latBinEdges="39.0, 40.5"
        pdfBinEdges="-3.0, -1.0, 1.0, 3.0"
        tectonicRegionTypes="Active Shallow Crust">
\textcolor{red}{    <disaggMatrix type="Mag" dims="3" poE="0.1" }
\textcolor{red}{            iml="0.033424622602">}
      <prob index="0" value="0.987374744394"/>
      <prob index="1" value="0.704295394366"/>
      <prob index="2" value="0.0802318409498"/>
\textcolor{red}{    </disaggMatrix>}
\textcolor{red}{    <disaggMatrix type="Dist" dims="4" poE="0.1" }
\textcolor{red}{            iml="0.033424622602">}
      <prob index="0" value="0.700851969171"/>
      <prob index="1" value="0.936680387051"/>
      <prob index="2" value="0.761883595568"/>
      <prob index="3" value="0.238687565571"/>
\textcolor{red}{    </disaggMatrix>}
\textcolor{red}{    <disaggMatrix type="TRT" dims="1" poE="0.1" }
\textcolor{red}{            iml="0.033424622602">}
      <prob index="0" value="0.996566187011"/>
\textcolor{red}{    </disaggMatrix>}
\textcolor{red}{    <disaggMatrix type="Mag,Dist" dims="3,4" poE="0.1" }
\textcolor{red}{            iml="0.033424622602">}
      <prob index="2,3" value="0.0"/>
\textcolor{red}{    </disaggMatrix>}
\textcolor{red}{    <disaggMatrix type="Mag,Dist,pdf" dims="3,4,3" poE="0.1" }
\textcolor{red}{            iml="0.033424622602">}
      <prob index="0,0,0" value="0.0785857271425"/>
      ...
\textcolor{red}{    </disaggMatrix>}
\textcolor{red}{    <disaggMatrix type="Lon,Lat" dims="2,1" poE="0.1"}
\textcolor{red}{            iml="0.033424622602">}
      <prob index="0,0" value="0.996566187011"/>
      <prob index="1,0" value="0.0"/>
\textcolor{red}{    </disaggMatrix>}
\textcolor{red}{    <disaggMatrix type="Mag,Lon,Lat" dims="3,2,1" poE="0.1"}
\textcolor{red}{            iml="0.033424622602">}
      <prob index="0,0,0" value="0.987374744394"/>
      <prob index="0,1,0" value="0.0"/>
      <prob index="1,0,0" value="0.704295394366"/>
      <prob index="1,1,0" value="0.0"/>
      <prob index="2,0,0" value="0.0802318409498"/>
      <prob index="2,1,0" value="0.0"/>
\textcolor{red}{    </disaggMatrix>}
\textcolor{red}{    <disaggMatrix type="Lon,Lat,TRT" dims="2,1,1" poE="0.1"}
\textcolor{red}{            iml="0.033424622602">}
      <prob index="0,0,0" value="0.996566187011"/>
      <prob index="1,0,0" value="0.0"/>
\textcolor{red}{    </disaggMatrix>}
  </disaggMatrices>
</nrml>
\end{Verbatim}
  \caption{Example of different disaggregation matrices produced by oq-engine}
  \label{lst:output_disagg_matrix}
\end{listing}




\subsection{Outputs from Event Based PSHA}
\label{subsec:output_event_based_psha}
The Event Based PSHA calculator computes and stores stochastic event sets and
the corresponding ground motion fields.

This calculator can also produce hazard curves and hazard maps exactly in the
same way as done using the Classical PSHA calculator.

The inset below shows an example of the list of results provided by the
\gls{acr:oqe} at the end of an event-based PSHA calculation:

\begin{Verbatim}[frame=single, commandchars=\\\{\}, fontsize=\small]
user@ubuntu:~$ oq engine --lo <calc_id>
id | name
\textcolor{red}{10 | Ground Motion Fields}
11 | Hazard Curves
12 | Hazard Maps
13 | Realizations
\textcolor{blue}{14 | Earthquake Ruptures}
15 | Events
16 | Uniform Hazard Spectra
\end{Verbatim}


This list in the inset above contains a set of ruptures (in green)
and their corresponding sets of ground motion fields (in red).

Exporting the outputs from the ruptures will produce, for each
realisation, a NRML file containing a collection of ruptures.
Below is an example:

\begin{Verbatim}[frame=single, commandchars=\\\{\}, fontsize=\small]
<?xml version="1.0" encoding="UTF-8"?>
<nrml xmlns:gml="http://www.opengis.net/gml"
	  xmlns="http://openquake.org/xmlns/nrml/0.5">
  <stochasticEventSetCollection sourceModelTreePath="b1">
    \textcolor{red}{<stochasticEventSet id="12" investigationTime="50.0">}
\textcolor{green}{     <rupture id="col=00~ses=0001~rup=0049-00"}
\textcolor{green}{         magnitude="4.55" strike="90.0" dip="90.0"}
\textcolor{green}{         rake="90.0" tectonicRegion="Active Shallow Crust">}
\textcolor{green}{       <planarSurface>}
\textcolor{green}{         <topLeft lon="12.233903801" lat="43.256198599"}
\textcolor{green}{                  depth="11.3933265259"/>}
\textcolor{green}{         <topRight lon="12.263958243" lat="43.2562025344"}
\textcolor{green}{                  depth="11.3933265259"/>}
\textcolor{green}{         <bottomLeft lon="12.233903801" lat="43.256198599"}
\textcolor{green}{                  depth="12.6066734741"/>}
\textcolor{green}{         <bottomRight lon="12.263958243" lat="43.2562025344"}
\textcolor{green}{                  depth="12.6066734741"/>}
\textcolor{green}{       </planarSurface>}
\textcolor{green}{     </rupture>}
      <rupture id="col=00~ses=0001~rup=0121-00"
              magnitude="4.65" strike="135.0" dip="90.0"
              rake="90.0" tectonicRegion="Active Shallow Crust">
        <planarSurface>
          <topLeft lon="11.45858812" lat="42.7429056814"
                  depth="11.3208667302"/>
          <topRight lon="11.4822820715" lat="42.7256333907"
                  depth="11.3208667302"/>
          <bottomLeft lon="11.45858812" lat="42.7429056814"
                  depth="12.6791332698"/>
          <bottomRight lon="11.4822820715" lat="42.7256333907"
                  depth="12.6791332698"/>
        </planarSurface>
      </rupture>
    \textcolor{red}{</stochasticEventSet>}
  </stochasticEventSetCollection>
</nrml>
\end{Verbatim}


The text in red shows the part which describes the generated
stochastic event sets and the investigation time covered. Inside the
<SES> tag there is a list of integers (a single integer in this example)
which are unique IDs for the seismic events associated to the rupture.
In general a rupture can occur more than once and the number of events
is given by the multiplicity attribute (in this case 1).

The text in green emphasises the portion of the text used to describe a
rupture. The information provided describes entirely the geometry of the
rupture as well as its rupturing properties (e.g. rake, magnitude). The
rupture ID is an integer that represents each rupture uniquely: it should
not be confused with the event ID, because the multiplicity of a rupture
can be different than 1.

Exporting the outputs from the gmfs will produce an xml file for each
realisation containing the corresponding ground motion fields. Below is an
example of a gmf collection nrml file containing one ground motion field:

\begin{Verbatim}[frame=single, commandchars=\\\{\}, fontsize=\small]
<?xml version="1.0" encoding="UTF-8"?>
<nrml xmlns:gml="http://www.opengis.net/gml"
      xmlns="http://openquake.org/xmlns/nrml/0.5">
  <gmfCollection sourceModelTreePath="b1" gsimTreePath="b1">
    <gmfSet investigationTime="50.0" stochasticEventSetId="12">
      <gmf IMT="PGA" ruptureId="col=00~ses=0001~rup=0049-00">
        <node gmv="0.0105891230432" lon="11.1240023202"
            lat="43.5107462335"/>
        <node gmv="0.00905803920023" lon="11.1241875202"
            lat="43.6006783941"/>
        <node gmv="0.00637664420977" lon="11.1243735810"
            lat="43.6906105547"/>
        <node gmv="0.00476533134789" lon="11.1245605075"
            lat="43.7805427153"/>
        <node gmv="0.00452594698469" lon="11.1247483046"
            lat="43.8704748759"/>
        ...
        <node gmv="0.00017301076646" lon="11.3782630185"
            lat="44.5129482397"/>
      </gmf>
    </gmfSet>
  </gmfCollection>
</nrml>
\end{Verbatim}


The `sourcegroups` output produces a csv file listing the tectonic region
types involved in the calculation and the effective number of ruptures
generated by each of them. An example of such a file is shown below.

\begin{table}[htbp]
\centering
\begin{tabular}{llr}

\hline
\rowcolor{lightgray}
\bf{grp_id} & \bf{trt} & \bf{eff_ruptures} \\
\hline
0 & Active Shallow Crust & 283 \\
1 & Stable Shallow Crust & 24 \\
2 & Subduction Interface & 2 \\
\hline

\end{tabular}
\caption{Example of a source groups output file}
\label{output:event_based_sourcegroups}
\end{table}


\subsection{Outputs from Scenario Hazard Analysis}
\label{subsec:output_scenario_hazard}
By default, the scenario hazard calculator computes and stores
\glspl{acr:gmf} for each GMPE specified in the job configuration file. The
\glspl{acr:gmf} will be computed at each of the sites and for each of the
intensity measure types specified in the job configuration file.

Exporting the outputs from the \glspl{acr:gmf} in the xml format will produce
an xml file for each realisation containing the corresponding ground motion
fields. Listing~\ref{lst:output_gmf_scenario_xml} is an example of a \gls{acr:gmf}
collection NRML file containing one \gls{acr:gmf}:

\begin{listing}[htbp]
  \inputminted[firstline=1,firstnumber=1,fontsize=\footnotesize,frame=single,linenos,bgcolor=lightgray]{xml}{oqum/hazard/verbatim/output_gmf_scenario.xml}
  \caption{Example ground motion field collection output file for a scenario}
  \label{lst:output_gmf_scenario_xml}
\end{listing}

Exporting the outputs from the \glspl{acr:gmf} in the csv format results in
two csv files illustrated in the example files in
Table~\ref{output:gmf_scenario} and Table~\ref{output:sitemesh}. The sites csv
file provides the association between the site ids in the \glspl{acr:gmf} csv
file with their latitude and longitude coordinates.

\begin{table}[htbp]
\centering
\begin{tabular}{cccccc}

\hline
\rowcolor{lightgray}
\textbf{rlzi} & \textbf{sid} & \textbf{eid} & \textbf{gmv\_PGA} & \textbf{gmv\_SA(0.3)} & \textbf{gmv\_SA(1.0)} \\
\hline
0 & 0 & 0 & 0.062 & 0.119 & 0.157 \\
0 & 1 & 0 & 0.086 & 1.533 & 0.260 \\
0 & 2 & 0 & 0.223 & 1.647 & 0.232 \\
... & ... & ... & ... & ... & ... \\
1 & 4 & 99 & 2.467 & 0.750 & 1.918 \\
1 & 5 & 99 & 0.601 & 0.828 & 2.272 \\
1 & 6 & 99 & 0.514 & 0.340 & 1.202
\hline

\end{tabular}
\caption{Example of a ground motion fields csv output file for a scenario (\href{https://raw.githubusercontent.com/gem/oq-engine/master/doc/manual/oqum/hazard/verbatim/output_scenario_gmfs.csv}{Download example})}
\label{output:gmf_scenario}
\end{table}

In this example, the gmfs have been computed using two different GMPEs, so the
realization indices ('rlzi') in the first column of the example gmfs file are
either 0 or 1. The gmfs file lists the ground motion values for 100
simulations of the scenario, so the event indices ('eid') in the third column
go from 0–99. There are seven sites with indices 0–6 ('sid') which are
repeated in the second column for each of the 100 simulations of the event and
for each of the two GMPEs. Finally, the subsequent columns list the ground
motion values for each of the intensity measure types specified in the job
configuration file.

\begin{table}[htbp]
\centering
\begin{tabular}{ccc}

\hline
\rowcolor{lightgray}
\textbf{site\_id} & \textbf{lon} & \textbf{lat} \\
\hline
0 & -122.57000 & 38.11300 \\
1 & -122.11400 & 38.11300 \\
2 & -122.00000 & 37.91000 \\
3 & -122.00000 & 38.00000 \\
4 & -122.00000 & 38.11300 \\
5 & -122.00000 & 38.22500 \\
6 & -121.88600 & 38.11300 \\
\hline

\end{tabular} \caption{Example of a sites csv output file for a scenario (\href{https://raw.githubusercontent.com/gem/oq-engine/master/doc/manual/oqum/hazard/verbatim/output_scenario_sites.csv}{Download example})}
\label{output:sitemesh}
\end{table}

