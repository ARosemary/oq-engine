An example of disaggregation calculation is given considering a source model
consisting of two sources (area and simple fault) belonging to two different
tectonic region types. The calculation is defined with the following configuration
file (Listing~\ref{lst:config_disagg}):

\begin{listing}[htbp]
  \inputminted[firstline=1,firstnumber=1,fontsize=\footnotesize,frame=single,linenos,bgcolor=lightgray,label=job.ini]{ini}{oqum/hazard/verbatim/config_disagg.ini}
  \caption{Example configuration file for a disaggregation calculation (\href{https://raw.githubusercontent.com/gem/oq-engine/master/doc/manual/oqum/hazard/verbatim/config_disagg.ini}{Download example})}
  \label{lst:config_disagg}
\end{listing}

Disaggregation matrices are computed for a single site (located between the
two sources) for a ground motion value corresponding to a probability value
equal to 0.1 (\texttt{poes\_\-disagg = 0.1}). Magnitude values are classified
in one magnitude unit bins (\texttt{mag\_\-bin\_\-width = 1.0}), distances in
bins of 10 km (\texttt{distance\_\-bin\_\-width = 10.0}), coordinates in bins
of 0.2 degrees (\texttt{coordinate\_\-bin\_\-width = 0.2}). 3 epsilons bins
are considered (\texttt{num\_\-epsilon\_\-bins = 3}).
