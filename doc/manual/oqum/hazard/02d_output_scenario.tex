By default, the scenario hazard calculator computes and stores
\glspl{acr:gmf} for each GMPE specified in the job configuration file. The
\glspl{acr:gmf} will be computed at each of the sites and for each of the
intensity measure types specified in the job configuration file.

Exporting the outputs from the \glspl{acr:gmf} in the csv format results in
two csv files illustrated in the example files in
Table~\ref{output:gmf_scenario} and Table~\ref{output:sitemesh}. The sites csv
file provides the association between the site ids in the \glspl{acr:gmf} csv
file with their latitude and longitude coordinates.

\begin{table}[htbp]
\centering
\begin{tabular}{cccccc}

\hline
\rowcolor{lightgray}
\textbf{rlzi} & \textbf{sid} & \textbf{eid} & \textbf{gmv\_PGA} & \textbf{gmv\_SA(0.3)} & \textbf{gmv\_SA(1.0)} \\
\hline
0 & 0 & 0 & 0.062 & 0.119 & 0.157 \\
0 & 1 & 0 & 0.086 & 1.533 & 0.260 \\
0 & 2 & 0 & 0.223 & 1.647 & 0.232 \\
... & ... & ... & ... & ... & ... \\
1 & 4 & 99 & 2.467 & 0.750 & 1.918 \\
1 & 5 & 99 & 0.601 & 0.828 & 2.272 \\
1 & 6 & 99 & 0.514 & 0.340 & 1.202
\hline

\end{tabular}
\caption{Example of a ground motion fields csv output file for a scenario (\href{https://raw.githubusercontent.com/gem/oq-engine/master/doc/manual/oqum/hazard/verbatim/output_scenario_gmfs.csv}{Download example})}
\label{output:gmf_scenario}
\end{table}

In this example, the gmfs have been computed using two different GMPEs, so the
realization indices ('rlzi') in the first column of the example gmfs file are
either 0 or 1. The gmfs file lists the ground motion values for 100
simulations of the scenario, so the event indices ('eid') in the third column
go from 0–99. There are seven sites with indices 0–6 ('sid') which are
repeated in the second column for each of the 100 simulations of the event and
for each of the two GMPEs. Finally, the subsequent columns list the ground
motion values for each of the intensity measure types specified in the job
configuration file.

\begin{table}[htbp]
\centering
\begin{tabular}{ccc}

\hline
\rowcolor{lightgray}
\textbf{site\_id} & \textbf{lon} & \textbf{lat} \\
\hline
0 & -122.57000 & 38.11300 \\
1 & -122.11400 & 38.11300 \\
2 & -122.00000 & 37.91000 \\
3 & -122.00000 & 38.00000 \\
4 & -122.00000 & 38.11300 \\
5 & -122.00000 & 38.22500 \\
6 & -121.88600 & 38.11300 \\
\hline

\end{tabular} \caption{Example of a sites csv output file for a scenario (\href{https://raw.githubusercontent.com/gem/oq-engine/master/doc/manual/oqum/hazard/verbatim/output_scenario_sites.csv}{Download example})}
\label{output:sitemesh}
\end{table}
