Probabilistic risk calculations can be run using either the Classical
Probabilistic Risk Calculator or the Stochastic Event-Based Probabilistic Risk
Calculator. The following set of outputs is generated by both calculators:

\begin{enumerate}

  \item \Verb+loss_curves+: loss exceedance curves describe the probabilities
    of exceeding a set of loss ratios or loss values, within a given time span
    (or investigation interval).

  \item \Verb+loss_maps+: loss maps describe the loss (\Verb+value+) 
    that is exceeded at the selected probability of exceedance (\Verb+poE+)
    within the specified time period for all \glspl{asset} at each of the unique
    locations in the \gls{exposuremodel}.

  \item \Verb+avg_losses+: the average losses output describes the expected
    loss (\Verb+value+) within the time period specified by
    \Verb+risk_investigation_time+ for all \glspl{asset}
    in the \gls{exposuremodel}.

\end{enumerate}

In addition, with the Stochastic Event-Based Probabilistic Risk
Calculator, it is also possible to calculate the following types of outputs:

\begin{enumerate}

  \item \Verb+agg_curves+: aggregate loss curves describe the exceedance 
    probabilities for a set of loss values for the entire portfolio of 
    \glspl{asset} defined in the \gls{exposuremodel}.

  \item \Verb+event_loss_table+: an event loss table contains the aggregate 
    loss across all \glspl{asset} in the \gls{exposuremodel} for each of the
    simulated \glspl{rupture} in the \glsdesc{acr:ses}.

\end{enumerate}



\subsection{Loss exceedance curves}
\label{subsec:loss_curves}

Loss exceedance curves describe the probabilities of exceeding a set of loss
ratios or loss values, within a given time span (or investigation interval).
Depending upon the type of calculator used and the options defined before
running a probabilistic risk calculation, one or more of the sets of loss
exceedance curves described in the following subsections will be generated for
all loss types (amongst ``structural'', ``nonstructural'', ``contents'',
``occupants'', or ``business\_interruption'') for which a vulnerability model
file was provided in the configuration file.

\subsubsection{Asset loss exceedance curves}
\label{subsubsec:asset_loss_curves}

Individual asset loss exceedance curves for ground-up losses are always
generated for the Classical Probabilistic Risk Calculator. On the other hand,
individual asset loss exceedance curves are not generated for the Stochastic
Event-Based Probabilistic Risk Calculator. These results are stored in a comma
separate value (.csv) file as illustrated in the example shown in
Table~\ref{output:loss_curve_asset}.

\begin{table}[htbp]
\centering
\begin{tabular}{llll}

\hline
\rowcolor{lightgray}
\textbf{asset} & \textbf{loss\_type} & \textbf{loss} & \textbf{poe} \\
\hline
asset & loss\_type & loss  & poe      \\
a1    & structural & 0     & 1.00E+00 \\
a1    & structural & 100   & 1.00E+00 \\
a1    & structural & 400   & 8.43E-01 \\
a1    & structural & 1000  & 4.70E-01 \\
a1    & structural & 2000  & 1.78E-01 \\
a1    & structural & 3300  & 7.31E-02 \\
a1    & structural & 5000  & 3.30E-02 \\
a1    & structural & 6700  & 1.68E-02 \\
a1    & structural & 8000  & 1.01E-02 \\
a1    & structural & 9000  & 6.62E-03 \\
a1    & structural & 9600  & 4.95E-03 \\
a1    & structural & 9900  & 4.12E-03 \\
a1    & structural & 10000 & 3.86E-03 \\
\hline

\end{tabular}
\caption{Example of an asset loss curve output file}
\label{output:loss_curve_asset}
\end{table}

\subsubsection{Mean loss exceedance curves}
\label{subsubsec:mean_loss_curves}

For calculations involving multiple hazard branches, mean asset loss
exceedance curves are also generated for both the Classical Probabilistic Risk
Calculator and the Stochastic Event-Based Probabilistic Risk Calculator (if
the parameter ``loss\_ratios'' is defined in the configuration file). The
structure of the file is identical to that of the individual asset loss
exceedance curve output file.

\subsubsection{Quantile loss exceedance curves}
\label{subsubsec:quantile_loss_curves}

For calculations involving multiple hazard branches, quantile asset loss
exceedance curves can also be generated for both the Classical Probabilistic
Risk Calculator and the Stochastic Event-Based Probabilistic Risk Calculator
(if the parameter ``loss\_ratios'' is defined in the configuration file). The
quantiles for which loss curves will be calculated should have been defined in
the job configuration file for the calculation using the parameter
\Verb+quantiles+. The structure of the file is identical to that of
the individual asset loss exceedance curve output file.

\subsubsection{Aggregate loss exceedance curves}
\label{subsubsec:aggregate_loss_curves}

Aggregate loss exceedance curves are generated only by the Stochastic Event-
Based Probabilistic Risk Calculator and describe the probabilities of
exceedance of the total loss across the entire portfolio for a set of loss
values within a given time span (or investigation interval). These results are
exported in a comma separate value (.csv) file as illustrated in the example
shown in Table~\ref{output:loss_curve_aggregate}.

\begin{table}[htbp]
\centering
\begin{tabular}{ccrr}

\hline
\rowcolor{lightgray}
\textbf{annual\_frequency\_of\_exceedence} & \textbf{return\_period} & \textbf{structural} & \textbf{nonstructural} \\
\hline
1E+00 & 1 & - & 1,440.07 \\
5E-01 & 2 & 246.95 & 2,122.25 \\
2E-01 & 5 & 506.42 & 2,714.08 \\
1E-01 & 10 & 740.06 & 3,226.47 \\
5E-02 & 20 & 1,040.54 & 4,017.06 \\
2E-02 & 50 & 1,779.61 & 6,610.49 \\
1E-02 & 100 & 2,637.58 & 9,903.82 \\
5E-03 & 200 & 3,742.73 & 14,367.00 \\
2E-03 & 500 & 5,763.20 & 21,946.50 \\
1E-03 & 1,000 & 7,426.77 & 25,161.00 \\
5E-04 & 2,000 & 9,452.61 & 28,937.50 \\
2E-04 & 5,000 & 12,021.00 & 35,762.20 \\
1E-04 & 10,000 & 14,057.90 & 38,996.60 \\
\dots & \dots & \dots & \dots \\
\hline

\end{tabular}
\caption{Example of an aggregate loss curve}
\label{output:loss_curve_aggregate}
\end{table}


Same as described previously for individual assets, mean aggregate
loss exceedance curves and quantile aggregate loss exceedance curves
will also be generated when relevant.


\subsection{Probabilistic loss maps}
\label{subsec:probabilistic_loss_map}

A probabilistic loss map contains the losses that have a specified probability
of exceedance within a given time span (or investigation interval) throughout
the region of interest. This result can be generated using either the
Stochastic Event-Based Probabilistic Risk Calculator or the Classical
Probabilistic Risk Calculator.

The file snippet included in Table~\ref{output:probabilistic_loss_map}.
shows an example probabilistic loss map output file.

\begin{table}[htbp]
\centering
\begin{tabular}{ccccrr}

\hline
\rowcolor{lightgray}
\textbf{asset\_ref} & \textbf{taxonomy} & \textbf{lon} & \textbf{lat} & \textbf{structural$\sim$poe-0.02} & \textbf{structural$\sim$poe-0.1} \\
\hline
a1 & wood & -122.000 & 38.113 & 6,686.10 & 3,241.80 \\
a2 & concrete & -122.114 & 38.113 & 597.59 & 328.07 \\
a3 & wood & -122.570 & 38.113 & 251.73 & 136.64 \\
a4 & steel & -122.000 & 38.000 & 3,196.66 & 1,610.98 \\
a5 & wood & -122.000 & 37.910 & 949.26 & 431.26 \\
a6 & concrete & -122.000 & 38.225 & 1,549.72 & 577.30 \\
a7 & wood & -121.886 & 38.113 & 1,213.54 & 677.16 \\
\hline

\end{tabular}
\caption{Example of a probabilistic loss map output file}
\label{output:probabilistic_loss_map}
\end{table}


\subsection{Stochastic event loss tables}

The Stochastic Event-Based Probabilistic Risk Calculator will also produce an
aggregate event loss table. Each row of this table contains the rupture id,
and aggregated loss (sum of the losses from the collection of assets within
the region of interest), for each event in the stochastic event sets. The
rupture id listed in this table is linked with the rupture ids listed in the
stochastic event sets files.

The file snippet included in Table~\ref{output:event_loss_table_aggregate}
shows an example stochastic event loss table output file.

\begin{table}[htbp]
\centering
\resizebox{\columnwidth}{!}{
\begin{tabular}{lclccccr}

\hline
\rowcolor{lightgray}
\bf{event\_tag} & \bf{year} & \bf{tectonic\_region\_type} & \bf{magnitude} & \bf{centroid\_lon} & \bf{centroid\_lat} & \bf{centroid\_depth} & \bf{structural} \\
\hline
grp=00~ses=0001~rup=297176-01 & 1 & Active Shallow Crust & 6.05 & -122.2 & 37.7 & 7.50 & 2.63933E+10 \\
grp=00~ses=0001~rup=92276-01 & 2 & Active Shallow Crust & 6.05 & -120.6 & 39.4 & 17.20 & 3.10979E+06 \\
grp=00~ses=0001~rup=342752-01 & 2 & Active Shallow Crust & 5.55 & -121.4 & 36.8 & 2.15 & 7.87414E+07 \\
grp=00~ses=0001~rup=374635-01 & 4 & Active Shallow Crust & 5.65 & -121.4 & 37.1 & 12.50 & 1.25105E+08 \\
grp=00~ses=0001~rup=145877-01 & 5 & Active Shallow Crust & 5.25 & -122.9 & 38.9 & 21.00 & 6.30566E+06 \\
grp=00~ses=0001~rup=506902-01 & 6 & Active Shallow Crust & 5.95 & -120.5 & 39.7 & 9.90 & 5.80147E+05 \\
\dots & \dots & \dots & \dots & \dots & \dots & \dots & \dots \\
\hline

\end{tabular}
}
\caption{Example aggregate event loss table}
\label{output:event_loss_table_aggregate}
\end{table}


Asset event loss tables provide calculated losses for each of the
assets in the exposure model, for each event within the stochastic
event sets. Considering that the amount of data usually contained in
an asset event loss table is substantial, this table is not generated
by default and even when it is generated it cannot be exported: it can
only be accessed programmatically from the datastore. It is there for
debugging purposes only.
