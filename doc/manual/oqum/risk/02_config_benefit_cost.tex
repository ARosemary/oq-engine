As previously explained, this calculator uses loss exceedance curves which are
calculated using the Classical Probabilistic risk calculator. In order to run
this calculator, the parameter \Verb+calculation_mode+ needs to be set to
\Verb+classical_bcr+.

Most of the job configuration parameters required for running a classical
retrofit benefit-cost ratio calculation are the same as those described in the
previous section for the classical probabilistic risk calculator. The
remaining parameters specific to the classical retrofit benefit-cost ratio
calculator are illustrated through the examples below.

\paragraph{Example 1}

This example illustrates a classical probabilistic retrofit benefit-cost ratio
calculation which uses a single configuration file to first compute the hazard
curves for the given source model and ground motion model, then calculate loss
exceedance curves based on the hazard curves using both the original
vulnerability model and the vulnerability model for the retrofitted
structures, then calculate the reduction in average annual losses due to the
retrofits, and finally calculate the benefit-cost ratio for each asset. A
minimal job configuration file required for running a classical probabilistic
retrofit benefit-cost ratio calculation is shown in
Listing~\ref{lst:config_classical_bcr_combined}.

\begin{listing}[htbp]
  \inputminted[firstline=1,firstnumber=1,fontsize=\footnotesize,frame=single,linenos,bgcolor=lightgray,label=job.ini]{ini}{oqum/risk/verbatim/config_classical_bcr_combined.ini}
  \caption{Example configuration file for a classical probabilistic retrofit benefit-cost ratio calculation (\href{https://raw.githubusercontent.com/gem/oq-engine/master/doc/manual/oqum/risk/verbatim/config_classical_bcr_combined.ini}{Download example})}
  \label{lst:config_classical_bcr_combined}
\end{listing}

The new parameters introduced in the above example configuration file are
described below:

\begin{itemize}

  \item \Verb+vulnerability_retrofitted_file+: this parameter is used to
    specify the path to the \gls{vulnerabilitymodel} file containing the
    \glspl{vulnerabilityfunction} for the retrofitted asset

  \item \Verb+interest_rate+: this parameter is used in the calculation of the
    present value of potential future benefits by discounting future cash flows

  \item \Verb+asset_life_expectancy+: this variable defines the life
    expectancy or design life of the assets, and is used as the time-frame in
    which the costs and benefits of the retrofit will be compared

\end{itemize}

The above calculation can be run using the command line:

\begin{minted}[fontsize=\footnotesize,frame=single,bgcolor=lightgray]{shell-session}
user@ubuntu:~\$ oq engine --run job.ini
\end{minted}

After the calculation is completed, a message similar to the following will be
displayed:

\begin{minted}[fontsize=\footnotesize,frame=single,bgcolor=lightgray]{shell-session}
Calculation 2776 completed in 25 seconds. Results:
  id | name
5422 | Benefit-cost ratio distribution | BCR Map. type=structural, hazard=5420
\end{minted}