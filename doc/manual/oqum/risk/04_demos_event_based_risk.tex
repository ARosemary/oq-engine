This demo uses the same probabilistic seismic hazard assessment (PSHA) model
described in the previous examples in Section~\ref{sec:demos_classical_damage}
and Section~\ref{sec:demos_classical_risk}. However, instead of hazard curves,
sets of ground motion fields will be generated by the hazard calculation of
this demo. Again, since there is only one branch in the logic tree, only one
set of ground motion fields will be used in the risk calculations. The hazard
and risk jobs are defined in a single configuration file for this demo. To
trigger the hazard and risk calculations the following command needs to be
used:

\begin{minted}[fontsize=\footnotesize,frame=single,bgcolor=lightgray]{shell-session}
user@ubuntu:~\$ oq engine --run job.ini
\end{minted}

and the following results are expected:

\begin{minted}[fontsize=\footnotesize,frame=single,bgcolor=lightgray]{shell-session}
Calculation 8974 completed in 229 seconds. Results:
  id | name
1820 | Aggregate Loss Curves
1821 | Aggregate Loss Curves Statistics
1822 | Aggregate Loss Table
1823 | Average Asset Losses
1824 | Average Asset Loss Statistics
1826 | Asset Loss Maps
1827 | Asset Loss Maps Statistics
1828 | Realizations
\end{minted}
