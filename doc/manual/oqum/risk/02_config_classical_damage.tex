In order to run this calculator, the parameter \Verb+calculation_mode+ needs
to be set to \Verb+classical_damage+.

Most of the job configuration parameters required for running a classical
probabilistic damage calculation are the same as those described in the
section for the scenario damage calculator. The remaining parameters specific
to the classical probabilistic damage calculator are illustrated through the
examples below.

\paragraph{Example 1}

This example illustrates a classical probabilistic damage calculation which
uses a single configuration file to first compute the hazard curves for the
given source model and ground motion model and then calculate damage
distribution statistics based on the hazard curves. A minimal job
configuration file required for running a classical probabilistic damage
calculation is shown in Listing~\ref{lst:config_classical_damage_combined}.

\begin{listing}[htbp]
  \inputminted[firstline=1,firstnumber=1,fontsize=\footnotesize,frame=single,linenos,bgcolor=lightgray,label=job.ini]{ini}{oqum/risk/verbatim/config_classical_damage_combined.ini}
  \caption{Example combined configuration file for a classical probabilistic damage calculation (\href{https://raw.githubusercontent.com/gem/oq-engine/master/doc/manual/oqum/risk/verbatim/config_classical_damage_combined.ini}{Download example})}
  \label{lst:config_classical_damage_combined}
\end{listing}

The general parameters \Verb+description+ and \Verb+calculation_mode+, and
\Verb+exposure_file+ have already been described earlier in
Section~\ref{sec:config_scenario_damage}. The parameters related to the
hazard curves computation have been described earlier in
Section~\ref{subsec:config_classical_psha}.

In this case, the hazard curves will be computed at each of the locations of
the assets in the exposure model, for each of the intensity measure types
found in the provided set of \glspl{fragilitymodel}. The above calculation can be
run using the command line:

\begin{minted}[fontsize=\footnotesize,frame=single,bgcolor=lightgray]{shell-session}
user@ubuntu:~\$ oq engine --run job.ini
\end{minted}

After the calculation is completed, a message similar to the following will be
displayed:

\begin{minted}[fontsize=\footnotesize,frame=single,bgcolor=lightgray]{shell-session}
Calculation 2741 completed in 12 seconds. Results:
  id | name
5359 | Asset Damage Distribution
\end{minted}


\paragraph{Example 2}

This example illustrates a classical probabilistic damage calculation which
uses separate configuration files for the hazard and risk parts of a classical
probabilistic damage assessment. The first configuration file shown in
Listing~\ref{lst:config_classical_damage_hazard} contains input models and parameters
required for the computation of the hazard curves. The second configuration
file shown in Listing~\ref{lst:config_classical_damage} contains input models
and parameters required for the calculation of the probabilistic damage
distribution for a portfolio of assets based on the hazard curves and
fragility models.

\begin{listing}[htbp]
  \inputminted[firstline=1,firstnumber=1,fontsize=\footnotesize,frame=single,linenos,bgcolor=lightgray,label=job\_hazard.ini]{ini}{oqum/risk/verbatim/config_classical_hazard.ini}
  \caption{Example hazard configuration file for a classical probabilistic damage calculation (\href{https://raw.githubusercontent.com/gem/oq-engine/master/doc/manual/oqum/risk/verbatim/config_classical_hazard.ini}{Download example})}
  \label{lst:config_classical_damage_hazard}
\end{listing}

\begin{listing}[htbp]
  \inputminted[firstline=1,firstnumber=1,fontsize=\footnotesize,frame=single,linenos,bgcolor=lightgray,label=job\_damage.ini]{ini}{oqum/risk/verbatim/config_classical_damage.ini}
  \caption{Example risk configuration file for a classical probabilistic damage calculation (\href{https://raw.githubusercontent.com/gem/oq-engine/master/doc/manual/oqum/risk/verbatim/config_classical_damage.ini}{Download example})}
  \label{lst:config_classical_damage}
\end{listing}

Now, the above calculations described by the two configuration files
``job\_hazard.ini'' and ``job\_damage.ini'' can be run sequentially or
separately, as illustrated in Example~2 in
Section~\ref{sec:config_scenario_damage}. The new parameters introduced in the
above example configuration file are described below:

\begin{itemize}

  \item \Verb+risk_investigation_time+: an optional parameter that can be used
    in probabilistic damage or risk calculations where the period of interest
    for the risk calculation is different from the period of interest for the 
    hazard calculation. If this parameter is not explicitly set, the 
    \glsdesc{acr:oqe} will assume that the risk calculation is over the same 
    time period as the preceding hazard calculation.

  \item \Verb+steps_per_interval+: an optional parameter that can be used to
    specify whether discrete fragility functions in the fragility models should
    be discretized further, and if so, how many intermediate steps to use for
    the discretization. Setting 

    steps\_per\_interval = n

    will result in the \glsdesc{acr:oqe} discretizing the discrete fragility
    models using (n - 1) linear interpolation steps between each pair of 
    {intensity level, poe} points.

    The default value of this parameter is one, implying no interpolation.

\end{itemize}
