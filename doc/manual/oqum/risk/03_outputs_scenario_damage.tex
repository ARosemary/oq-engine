The Scenario Damage Calculator produces the following set of output files for
all loss types (amongst ``structural'', ``nonstructural'', ``contents'', or
``business\_interruption'') for which a fragility model file was provided in
the configuration file:

\begin{enumerate}

  \item \Verb+dmg_dist_per_asset+: this file contains the damage distribution
    statistics for each of the individual \glspl{asset} defined in the
    \gls{exposuremodel} that fall within the \Verb+region_constraint+ and have
    a computed \gls{acr:gmf} value available within the defined
    \Verb+asset_hazard_distance+. For each \gls{asset}, the mean number of
    buildings (\Verb+mean+) and associated standard deviation (\Verb+stddev+)
    of the number of buildings in each damage state are listed in this file.

  \item \Verb+dmg_dist_per_taxonomy+: this file contains the aggregated damage
    distribution statistics for each of the \glspl{taxonomy} defined in the
    \gls{exposuremodel}. For each \gls{taxonomy}, the mean number of
    buildings (\Verb+mean+) and associated standard deviation (\Verb+stddev+)
    of the number of buildings in each damage state are listed in this file.

  \item \Verb+dmg_dist_total+: this file contains the aggregated damage
    distribution statistics for the entire portfolio of \glspl{asset} defined
    in the \gls{exposuremodel}. The mean (\Verb+mean+) and associated standard
    deviation (\Verb+stddev+) of the total number of buildings in each
    damage state are listed in this file.

  \item \Verb+collapse_map+: this file contains mean (\Verb+mean+) and
    associated standard deviation (\Verb+stddev+) of the number of buildings
    in the ultimate limit state for all \glspl{asset} at each of the unique
    locations in the \gls{exposuremodel}.

\end{enumerate}

In addition to the above output files which are produced for all Scenario
Damage calculations, the following set of output files for all loss types
(amongst ``structural'', ``nonstructural'', ``contents'', or
``business\_interruption'') for which a \gls{consequencemodel} file was also
provided in the configuration file:

\begin{enumerate}
\setcounter{enumi}{4}

  \item \Verb+losses_by_asset+: this file contains the scenario consequence
    statistics for each of the individual \glspl{asset} defined in the
    \gls{exposuremodel} that fall within the \Verb+region_constraint+ and have
    a computed \gls{acr:gmf} value available within the defined
    \Verb+asset_hazard_distance+. For each \gls{asset}, the mean consequences
    (\Verb+mean+) and associated standard deviation (\Verb+stddev+) are listed
    in this file.

  \item \Verb+losses_by_tag+: this file contains the aggregated scenario
    consequence statistics for each of the \glspl{tags} defined in the
    \gls{exposuremodel}. For each \gls{tags}, the mean consequences
    (\Verb+mean+) and associated standard deviation (\Verb+stddev+) are listed
    in this file.

  \item \Verb+losses_total+: this file contains the aggregated scenario
    consequence statistics for the entire portfolio of \glspl{asset} defined
    in the \gls{exposuremodel}. The mean consequences (\Verb+mean+) and 
    associated standard deviation (\Verb+stddev+) are listed in this file.

\end{enumerate}

If the calculation involves multiple \glspl{acr:gmpe} as described in
Example~4 in Section~\ref{sec:config_scenario_damage}, separate output files
are generated for each of the above outputs, for each of the different
\glspl{acr:gmpe} used in the calculation.

These different output files for Scenario Damage calculations are described in
more detail in the following subsections.


\subsection{Scenario damage statistics}
\label{subsec:scenario_damage_statistics}

\subsubsection{Asset damage statistics}
\label{subsubsec:scenario_asset_damage_statistics}

This output contains the damage distribution statistics for each of the
individual \glspl{asset} defined in the \gls{exposuremodel} that fall within
the \Verb+region_constraint+ and have a computed \gls{acr:gmf} value available
within the defined \Verb+asset_hazard_distance+. An example output file for
structural damage is shown in the file snippet in Listing~\ref{lst:output_scenario_damage_asset}.

\begin{listing}[htbp]
  \inputminted[firstline=1,firstnumber=1,fontsize=\footnotesize,frame=single,bgcolor=lightgray]{xml}{oqum/risk/verbatim/output_scenario_damage_asset.xml}
  \caption{Example scenario damage statistics per asset}
  \label{lst:output_scenario_damage_asset}
\end{listing}

The key fields in the above output file are the following:

\begin{itemize}

  \item \Verb+damageStates+: this field serves the purposes of storing the set
    of damage states, as defined in the \gls{fragilitymodel} employed in the
    calculations

  \item \Verb+DDNode+: this attribute is used to store the damage distribution
    of a number of \glspl{asset}, at a given location (defined within the
    attribute \Verb+gml:Point+). For each \gls{asset}, the mean number of
    buildings (\Verb+mean+) and associated standard deviation (\Verb+stddev+)
    of the number of buildings in each damage state are listed.

\end{itemize}


\subsubsection{Taxonomy damage statistics}
\label{subsubsec:scenario_taxonomy_damage_statistics}

The Scenario Damage calculator also estimates the expected total number of
buildings of a certain \gls{taxonomy} in each damage state. This distribution
of damage per building \gls{taxonomy} is depicted in the example
output file snippet in Listing~\ref{lst:output_scenario_damage_taxonomy}.

\begin{listing}[htbp]
  \inputminted[firstline=1,firstnumber=1,fontsize=\footnotesize,frame=single,linenos,bgcolor=lightgray]{xml}{oqum/risk/verbatim/output_scenario_damage_taxonomy.xml}
  \caption{Example scenario damage statistics per taxonomy}
  \label{lst:output_scenario_damage_taxonomy}
\end{listing}

In the damage distribution per \gls{taxonomy}, each \Verb+DDNode+ contains the
statistics of the number of buildings in each damage state, belonging to a
given building class as specified by the \Verb+taxonomy+ attribute.


\subsubsection{Total damage statistics}
\label{subsubsec:scenario_total_damage_statistics}

Finally, a total damage distribution output file is also generated, which
contains the mean and standard deviation of the total number of buildings in
each damage state, as illustrated in the example file in
Listing~\ref{lst:output_scenario_damage_total}.

\begin{listing}[htbp]
  \inputminted[firstline=1,firstnumber=1,fontsize=\footnotesize,frame=single,linenos,bgcolor=lightgray]{xml}{oqum/risk/verbatim/output_scenario_damage_total.xml}
  \caption{Example total damage statistics}
  \label{lst:output_scenario_damage_total}
\end{listing}


\subsection{Scenario collapse maps}
\label{subsec:scenario_collapse_map}

Collapse maps are part of the Scenario Damage calculator outputs. These
results provide the spatial distribution of the number of the buildings in the
ultimate limit state throughout the area of interest. An example snippet from
an output file depicting a collapse map for structural damage is presented
below.

\begin{listing}[htbp]
  \inputminted[firstline=1,firstnumber=1,fontsize=\footnotesize,frame=single,bgcolor=lightgray]{xml}{oqum/risk/verbatim/output_scenario_damage_collapse.xml}
  \caption{Example collapse map}
  \label{lst:output_scenario_damage_collapse}
\end{listing}

The results for a number of \glspl{asset} at a given location are stored
within the field \Verb+CMNode+. This field is associated with a location
(defined within the \Verb+gml:Point+ attribute) and it contains the mean
number of buildings in the ultimate limit state (\Verb+mean+) and the
corresponding standard deviation (\Verb+stdDev+) for each \gls{asset}
(identified by the parameter \Verb+assetRef+).


\subsection{Scenario consequence statistics}
\label{subsec:scenario_consequence_statistics}

\subsubsection{Asset consequence statistics}
\label{subsubsec:scenario_asset_consequence_statistics}

This output contains the consequences statistics for each of the individual
\glspl{asset} defined in the \gls{exposuremodel} that fall within the
\Verb+region_constraint+ and have a computed \gls{acr:gmf} value available
within the defined \Verb+asset_hazard_distance+. An example output file for
structural damage consequences is shown in
Table~\ref{output:scenario_consequence_asset}.

\begin{table}[htbp]
\centering
\begin{tabular}{lrrrr}

\hline
\rowcolor{lightgray}
\bf{asset\_ref} & \bf{lon} & \bf{lat} & \bf{nonstructural-mean} & \bf{nonstructural-stddev} \\
\hline
a3 & -122.57000 & 38.11300 & 428.29 & 281.49 \\
a2 & -122.11400 & 38.11300 & 1220.84 & 1111.4 \\
a5 & -122.00000 & 37.91000 & 1390.59 & 859.10 \\
a4 & -122.00000 & 38.00000 & 2889.04 & 1663.33 \\
a1 & -122.00000 & 38.11300 & 3191.30 & 1707.41 \\
a6 & -122.00000 & 38.22500 & 3310.62 & 2069.87 \\
a7 & -121.88600 & 38.11300 & 1415.19 & 845.83 \\
\hline

\end{tabular}
\caption{Example of a scenario asset consequences output file}
\label{output:scenario_consequence_asset}
\end{table}

The output file lists consequence statistics for all loss types (amongst
``structural'', ``nonstructural'', ``contents'', or
``business\_interruption'') for which a \gls{consequencemodel} file was also
provided in the configuration file in addition to the corresponding
\gls{fragilitymodel} file.


\subsubsection{Taxonomy consequence statistics}
\label{subsubsec:scenario_taxonomy_consequence_statistics}

The Scenario Damage calculator also estimates the expected total consequences
for buildings of a certain \gls{taxonomy}, as depicted in the sample output
shown in Table~\ref{output:scenario_consequence_taxonomy}.

\begin{table}[htbp]
\centering
\begin{tabular}{lrrrr}

\hline
\rowcolor{lightgray}
\bf{taxonomy} & \bf{contents-mean} & \bf{contents-stddev} & \bf{structural-mean} & \bf{structural-stddev} \\
\hline
concrete & 6425.37 & 2965.17 & 1338.30 & 1661.46 \\
steel & 4531.46 & 2821.97 & 597.90 & 743.71 \\
wood & 2889.04 & 1663.33 & 334.09 & 386.80 \\
\hline

\end{tabular}
\caption{Example of a scenario taxonomy consequences output file}
\label{output:scenario_consequence_taxonomy}
\end{table}


\subsubsection{Total consequence statistics}
\label{subsubsec:scenario_total_consequence_statistics}

Finally, a total consequences output file is also generated, which contains
the mean and standard deviation of the total consequences for the selected
scenario, as illustrated in the example shown in
Table~\ref{output:scenario_consequence_total}.

\begin{table}[htbp]
\centering
\begin{tabular}{rrrr}

\hline
\rowcolor{lightgray}
\bf{contents-mean} & \bf{contents-stddev} & \bf{structural-mean} & \bf{structural-stddev} \\
\hline
13845.87 & 6517.61 & 2270.29 & 2440.90 \\
\hline

\end{tabular}
\caption{Example of a scenario total consequences output file}
\label{output:scenario_consequence_total}
\end{table}
