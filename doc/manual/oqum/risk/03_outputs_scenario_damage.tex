The Scenario Damage Calculator produces the following output file for
all loss types (amongst ``structural'', ``nonstructural'', ``contents'', or
``business\_interruption'') for which a fragility model file was provided in
the configuration file:

\begin{enumerate}

  \item \Verb+Average Asset Damages+: this file contains the damage distribution
    statistics for each of the individual \glspl{asset} defined in the
    \gls{exposuremodel} that fall within the \Verb+region_constraint+ and have
    a computed \gls{acr:gmf} value available within the defined
    \Verb+asset_hazard_distance+. For each \gls{asset}, the mean number of
    buildings (\Verb+mean+) and associated standard deviation (\Verb+stddev+)
    of the number of buildings in each damage state are listed in this file.

\end{enumerate}

In addition, if the OpenQuake-QGIS \gls{acr:irmt} plugin is used for
visualizing or exporting the results from a Scenario Damage Calculation, the
following additional outputs can be exported:

\begin{enumerate}
\setcounter{enumi}{1}

  \item \Verb+dmg_by_tag+: this file contains the aggregated damage
    distribution statistics for each of the \glspl{tag} defined in the
    \gls{exposuremodel}. For each \gls{tag}, the mean number of
    buildings (\Verb+mean+) in each damage state are listed in this file.

  \item \Verb+dmg_total+: this file contains the aggregated damage
    distribution statistics for the entire portfolio of \glspl{asset} defined
    in the \gls{exposuremodel}. The mean (\Verb+mean+) and associated standard
    deviation (\Verb+stddev+) of the total number of buildings in each
    damage state are listed in this file.

\end{enumerate}

In addition to the above asset-level damage output file which is
produced for all Scenario Damage calculations, the following output file is
also produced for all loss types
(amongst ``structural'', ``nonstructural'', ``contents'', or
``business\_interruption'') for which a \gls{consequencemodel} file was also
provided in the configuration file:

\begin{enumerate}
\setcounter{enumi}{3}

  \item \Verb+Average Asset Losses+: this file contains the scenario consequence
    statistics for each of the individual \glspl{asset} defined in the
    \gls{exposuremodel} that fall within the \Verb+region_constraint+ and have
    a computed \gls{acr:gmf} value available within the defined
    \Verb+asset_hazard_distance+. For each \gls{asset}, the mean consequences
    (\Verb+mean+) and associated standard deviation (\Verb+stddev+) are listed
    in this file.

\end{enumerate}

In addition, if the OpenQuake-QGIS \gls{acr:irmt} plugin is used for
visualizing or exporting the results from a Scenario Damage Calculation, the
following additional outputs can be exported:

\begin{enumerate}
\setcounter{enumi}{4}

  \item \Verb+losses_by_tag+: this file contains the aggregated scenario
    consequence statistics for each of the \glspl{tag} defined in the
    \gls{exposuremodel}. For each \gls{tag}, the mean consequences
    (\Verb+mean+) and associated standard deviation (\Verb+stddev+) are listed
    in this file.

  \item \Verb+losses_total+: this file contains the aggregated scenario
    consequence statistics for the entire portfolio of \glspl{asset} defined
    in the \gls{exposuremodel}. The mean consequences (\Verb+mean+) and 
    associated standard deviation (\Verb+stddev+) are listed in this file.

\end{enumerate}

If the calculation involves multiple \glspl{acr:gmpe} as described in
Example~4 in Section~\ref{sec:config_scenario_damage}, separate output files
are generated for each of the above outputs, for each of the different
\glspl{acr:gmpe} used in the calculation.

These different output files for Scenario Damage calculations are described in
more detail in the following subsections.


\subsection{Scenario damage statistics}
\label{subsec:scenario_damage_statistics}

\subsubsection{Asset damage statistics}
\label{subsubsec:scenario_asset_damage_statistics}

This output contains the damage distribution statistics for each of the
individual \glspl{asset} defined in the \gls{exposuremodel} that fall within
the \Verb+region_constraint+ and have a computed \gls{acr:gmf} value available
within the defined \Verb+asset_hazard_distance+. An example output file for
structural damage is shown in the file snippet in 
Table~\ref{output:scenario_damage_asset}.

\begin{table}[htbp]
\centering
\resizebox{\columnwidth}{!}{
\begin{tabular}{llccrrrrc}
\hline
\rowcolor{lightgray}
 & & & & \textbf{structural} & \textbf{structural} & \textbf{structural} & \textbf{structural} & \ldots \\
\rowcolor{lightgray}
\textbf{asset\_ref} & \textbf{taxonomy} & \textbf{lon} & \textbf{lat} & \textbf{ds0\_mean} & \textbf{ds0\_stdv} & \textbf{ds1\_mean} & \textbf{ds1\_stdv} & \ldots \\ \hline
a3 & tax1 & -122.57000 & 38.11300 & 1.00 & 0.00 & 0.00 & 0.00 & \ldots \\
a2 & tax2 & -122.11400 & 38.11300 & 0.80 & 0.19 & 0.10 & 0.07 & \ldots \\
a5 & tax1 & -122.00000 & 37.91000 & 0.93 & 0.18 & 0.06 & 0.16 & \ldots \\
a4 & tax3 & -122.00000 & 38.00000 & 0.46 & 0.28 & 0.15 & 0.06 & \ldots \\
a1 & tax1 & -122.00000 & 38.11300 & 0.42 & 0.42 & 0.36 & 0.29 & \ldots \\
a6 & tax2 & -122.00000 & 38.22500 & 0.44 & 0.26 & 0.17 & 0.05 & \ldots \\
a7 & tax1 & -121.88600 & 38.11300 & 0.93 & 0.21 & 0.06 & 0.17 & \ldots \\
\hline
\end{tabular}
}
\caption{Example of a scenario asset damage distribution output file}
\label{output:scenario_damage_asset}
\end{table}



The output file lists the mean and standard deviation of the number
of buildings in each damage state for each asset in the exposure model
for all loss types (amongst `structural'', ``nonstructural'', ``contents'', or
``business\_interruption'') for which a \gls{consequencemodel} file was also
provided in the configuration file in addition to the corresponding
\gls{fragilitymodel} file.


\subsubsection{Damage statistics by tag}
\label{subsubsec:scenario_tag_damage_statistics}

If the OpenQuake-QGIS \gls{acr:irmt} plugin is used for visualizing or
exporting the results, the Scenario Damage calculator can also estimate the
expected total number of buildings of a certain combination of \glspl{tag} in
each damage state and made available for export as a csv file. This
distribution of damage per building \gls{tag} is depicted in the example
output file snippet in Table~\ref{output:scenario_damage_tag}.

\begin{table}[htbp]
\centering
\begin{tabular}{lrrrrc}
\hline
\rowcolor{lightgray}
& \textbf{structural} & \textbf{structural} & \textbf{structural} & \textbf{structural} & \ldots \\
\rowcolor{lightgray}
\textbf{taxonomy} & \textbf{ds0\_mean} & \textbf{ds1\_mean} & \textbf{ds2\_mean} & \textbf{ds3\_mean} & \ldots \\ \hline
taxonomy=wood & 3,272.48 & 592.55 & 479.19 & 422.34 & \ldots \\
taxonomy=concrete & 1,241.94 & 389.94 & 272.69 & 91.63 & \ldots \\
taxonomy=steel & 460.72 & 279.44 & 152.18 & 57.43 & \ldots \\
\hline
\end{tabular}
\caption{Example of a scenario damage distribution per tag output file}
\label{output:scenario_damage_tag}
\end{table}



The output file lists the mean of the total number
of buildings in each damage state for each tag found in the
exposure model for all loss types (amongst
``structural'', ``nonstructural'', ``contents'', or
``business\_interruption'').


\subsubsection{Total damage statistics}
\label{subsubsec:scenario_total_damage_statistics}

Finally, a total damage distribution output file can also be generated if the
OpenQuake-QGIS \gls{acr:irmt} plugin is used for visualizing or exporting the
results from a Scenario Damage Calculation, which will contain the mean and
standard deviation of the total number of buildings in each damage state, as
illustrated in the example file in Table~\ref{output:scenario_damage_tag}.

\begin{table}[htbp]
\centering
\begin{tabular}{llr}
\hline
\rowcolor{lightgray}
\textbf{loss\_type} & \textbf{damage\_state} & \textbf{damage\_value} \\ \hline
structural & no\_damage\_mean & 4,975.13 \\
structural & no\_damage\_stdv & 767.36 \\
structural & ds1\_mean & 904.06 \\
structural & ds1\_stdv & 442.56 \\
structural & ds2\_mean & 564.35 \\
structural & ds2\_stdv & 207.71 \\
structural & ds3\_mean & 246.44 \\
structural & ds3\_stdv & 161.54 \\
structural & ds4\_mean & 310.03 \\
structural & ds4\_stdv & 231.92 \\
\hline
\end{tabular}
\caption{Example of a scenario total damage distribution output file}
\label{output:scenario_damage_total}
\end{table}




\subsection{Scenario consequence statistics}
\label{subsec:scenario_consequence_statistics}

\subsubsection{Asset consequence statistics}
\label{subsubsec:scenario_asset_consequence_statistics}

This output contains the consequences statistics for each of the individual
\glspl{asset} defined in the \gls{exposuremodel} that fall within the
\Verb+region_constraint+ and have a computed \gls{acr:gmf} value available
within the defined \Verb+asset_hazard_distance+. An example output file for
structural damage consequences is shown in
Table~\ref{output:scenario_consequence_asset}.

\begin{table}[htbp]
\centering
\begin{tabular}{lrrrr}

\hline
\rowcolor{lightgray}
\bf{asset\_ref} & \bf{lon} & \bf{lat} & \bf{nonstructural-mean} & \bf{nonstructural-stddev} \\
\hline
a3 & -122.57000 & 38.11300 & 428.29 & 281.49 \\
a2 & -122.11400 & 38.11300 & 1220.84 & 1111.4 \\
a5 & -122.00000 & 37.91000 & 1390.59 & 859.10 \\
a4 & -122.00000 & 38.00000 & 2889.04 & 1663.33 \\
a1 & -122.00000 & 38.11300 & 3191.30 & 1707.41 \\
a6 & -122.00000 & 38.22500 & 3310.62 & 2069.87 \\
a7 & -121.88600 & 38.11300 & 1415.19 & 845.83 \\
\hline

\end{tabular}
\caption{Example of a scenario asset consequences output file}
\label{output:scenario_consequence_asset}
\end{table}

The output file lists consequence statistics for all loss types (amongst
``structural'', ``nonstructural'', ``contents'', or
``business\_interruption'') for which a \gls{consequencemodel} file was also
provided in the configuration file in addition to the corresponding
\gls{fragilitymodel} file.


\subsubsection{Total consequence statistics}
\label{subsubsec:scenario_total_consequence_statistics}

Finally, if the OpenQuake-QGIS \gls{acr:irmt} plugin is used for visualizing
or exporting the results from a Scenario Damage Calculation, a total
consequences output file can also be generated, which will contain the mean and
standard deviation of the total consequences for the selected scenario, as
illustrated in the example shown in
Table~\ref{output:scenario_consequence_total}.

\begin{table}[htbp]
\centering
\begin{tabular}{rrrr}

\hline
\rowcolor{lightgray}
\bf{contents-mean} & \bf{contents-stddev} & \bf{structural-mean} & \bf{structural-stddev} \\
\hline
13845.87 & 6517.61 & 2270.29 & 2440.90 \\
\hline

\end{tabular}
\caption{Example of a scenario total consequences output file}
\label{output:scenario_consequence_total}
\end{table}
