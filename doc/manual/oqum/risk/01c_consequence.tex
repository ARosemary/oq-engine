Starting from \glsdesc{acr:oqe17}, the Scenario Damage calculator also accepts
\glspl{consequencemodel} in addition to \glspl{fragilitymodel}, in order to
estimate consequences based on the calculated damage distribution. The user
may provide one \gls{consequencemodel} file corresponding to each loss type
(amongst structural, nonstructural, contents, and business interruption) for
which a \gls{fragilitymodel} file is provided. Whereas providing a
\gls{fragilitymodel} file for at least one loss type is mandatory for running
a Scenario Damage calculation, providing corresponding \gls{consequencemodel}
files is optional.

This section describes the schema currently used to store
\glspl{consequencemodel}, which are optional inputs for the Scenario Damage
Calculator. A \gls{consequencemodel} defines a set of
\glspl{consequencefunction}, describing the distribution of the loss (or
consequence) ratio conditional on a set of discrete limit (or damage) states.
These \gls{consequencefunction} can be currently defined in \glsdesc{acr:oqe}
by specifying the parameters of the continuous distribution of the loss ratio
for each limit state specified in the fragility model for the corresponding
loss type, for each taxonomy defined in the exposure model.

An example \gls{consequencemodel} is shown in Listing~\ref{lst:input_consequence}.

\begin{listing}[htbp]
  \inputminted[firstline=1,firstnumber=1,fontsize=\footnotesize,frame=single,linenos,bgcolor=lightgray]{xml}{oqum/risk/verbatim/input_consequence.xml}
  \caption{Example consequence model (\href{https://raw.githubusercontent.com/gem/oq-engine/master/doc/manual/oqum/risk/verbatim/input_consequence.xml}{Download example})}
  \label{lst:input_consequence}
\end{listing}	

The initial portion of the schema contains general information that describes
some general aspects of the \gls{consequencemodel}. The information in this
metadata section is common to all of the functions in the
\gls{consequencemodel} and needs to be included at the beginning of every
\gls{consequencemodel} file. The parameters are described below:

\begin{itemize}

  \item \Verb+id+: a unique string used to identify the \gls{consequencemodel}.
    This string can contain letters~(a--z; A--Z), numbers~(0--9), dashes~(-), 
    and underscores~(\_), with a maximum of 100~characters.

  \item \Verb+assetCategory+: an optional string used to specify the type of
    \glspl{asset} for which \glspl{consequencefunction} will be defined in
    this file (e.g: buildings, lifelines).

  \item \Verb+lossCategory+: mandatory; valid strings for this attribute are 
    ``structural'', ``nonstructural'', ``contents'', and 
    ``business\_interruption''.

  \item \Verb+description+: mandatory; a brief string (ASCII) with further 
    information about the \gls{consequencemodel}, for example, which building
    typologies are covered or the source of the functions in the
    \gls{consequencemodel}.

  \item \Verb+limitStates+: mandatory; this field is used to define the number and 
    nomenclature of each limit state. Four limit states are employed in the 
    example above, but it is possible to use any number of discrete states. The 
    limit states must be provided as a set of strings separated by whitespaces 
    between each limit state. Each limit state string can contain
    letters~(a--z; A--Z), numbers~(0--9), dashes~(-), and underscores~(\_). 
    Please ensure that there is no whitespace within the name of any individual
    limit state. The number and nomenclature of the limit states used in the
    \gls{consequencemodel} should match those used in the corresponding
    \gls{fragilitymodel}.

\end{itemize}

\inputminted[firstline=4,firstnumber=4,lastline=9,fontsize=\footnotesize,frame=single,linenos,bgcolor=lightgray]{xml}{oqum/risk/verbatim/input_consequence.xml}

The following snippet from the above \gls{consequencemodel} example file
defines a \gls{consequencefunction} using a lognormal distribution to model
the uncertainty in the consequence ratio for each limit state:

\inputminted[firstline=11,firstnumber=11,lastline=16,fontsize=\footnotesize,frame=single,linenos,bgcolor=lightgray]{xml}{oqum/risk/verbatim/input_consequence.xml}

The following attributes are needed to define a \gls{consequencefunction}:

\begin{itemize}

  \item \Verb+id+: mandatory; a unique string used to identify the 
    \gls{taxonomy} for which the function is being defined. This string is used
    to relate the \gls{consequencefunction} with the relevant \gls{asset} in the 
    \gls{exposuremodel}. This string can contain letters~(a--z; A--Z),
    numbers~(0--9), dashes~(-), and underscores~(\_), with a maximum of
    100~characters.

  \item \Verb+dist+: mandatory; for vulnerability function which use a continuous 
    distribution to model the uncertainty in the conditional loss ratios, 
    this attribute should be set to either ``\Verb+LN+'' if using the lognormal
    distribution, or to ``\Verb+BT+'' if using the Beta distribution
    \footnote{Note that as of \glsdesc{acr:oqe18}, the uncertainty in the 
    consequence ratios is ignored, and only the mean consequence ratios for the
    set of limit states is considered when computing the consequences from the
    damage distribution. Consideration of the uncertainty in the consequence
    ratios is planned for future releases of the \glsdesc{acr:oqe}.}.

  \item \Verb+params+: mandatory; this field is used to define the parameters of 
    the continuous distribution used for modelling the uncertainty in the
    loss ratios for each limit state for this 
    \gls{consequencefunction}. For a lognormal distrbution, 
    the two parameters required to specify the function are the mean and 
    standard deviation of the consequence ratio. These parameters are defined for 
    each limit state using the attributes \Verb+mean+ and \Verb+stddev+ 
    respectively. The attribute \Verb+ls+ specifies the limit state for which 
    the parameters are being defined. The parameters for each limit state
    must be provided on a separate line. The number and names of the limit 
    states in each \gls{consequencefunction} must be equal to the number of limit 
    states defined in the corresponding \gls{fragilitymodel}
    using the attribute \Verb+limitStates+.

\end{itemize}