The parameter \Verb+calculation_mode+ needs to be set to
\Verb+event_based_risk+ in order to use this calculator.

Most of the job configuration parameters required for running a stochastic
event based risk calculation are the same as those described in the previous
sections for the scenario risk calculator and the classical probabilistic risk
calculator. The remaining parameters specific to the stochastic event based
risk calculator are illustrated through the example below.


\paragraph{Example 1}

This example illustrates a stochastic event based risk calculation which uses
a single configuration file to first compute the \glspl{acr:ses} and
\glspl{acr:gmf} for the given source model and ground motion model, and then
calculate event loss tables, loss exceedance curves and probabilistic
loss maps for structural losses, nonstructural losses and occupants,
based on the \glspl{acr:gmf}. The job configuration file required for
running this stochastic event based risk calculation is shown in
Listing~\ref{lst:config_event_based_risk_combined}.

\begin{listing}[htbp]
  \inputminted[firstline=1,firstnumber=1,fontsize=\scriptsize
  ,frame=single,bgcolor=lightgray,linenos,label=job.ini]{ini}{oqum/risk/verbatim/config_event_based_risk_combined.ini}
  \caption{Example combined configuration file for running a stochastic event based risk calculation (\href{https://raw.githubusercontent.com/gem/oq-engine/master/doc/manual/oqum/risk/verbatim/config_event_based_risk_combined.ini}{Download example})}
  \label{lst:config_event_based_risk_combined}
\end{listing}

Similar to that the procedure described for the Scenario Risk calculator, a
Monte Carlo sampling process is also employed in this calculator to take into
account the uncertainty in the conditional loss ratio at a particular
intensity level. Hence, the parameters \Verb+asset_correlation+ and
\Verb+master_seed+ may be defined as previously described for the Scenario
Risk calculator in Section~\ref{sec:config_scenario_risk}. The parameter
``risk\_investigation\_time'' specifies the time period for which the event
loss tables and loss exceedance curves will be calculated, similar to the
Classical Probabilistic Risk calculator. If this parameter is not provided in
the risk job configuration file, the time period used is the same as that
specifed in the hazard calculation using the parameter ``investigation\_time''.

The new parameters introduced in this example are described below:

\begin{itemize}

  \item \Verb+minimum_intensity+: this optional parameter specifies the minimum
    intensity levels for each of the intensity measure types in the risk model.
    Ground motion fields where each ground motion value is less than the 
    specified minimum threshold are discarded. This helps speed up calculations
    and reduce memory consumption by considering only those ground motion fields
    that are likely to contribute to losses. It is also possible to set the same
    threshold value for all intensity measure types by simply providing a single
    value to this parameter. For instance: ``minimum\_intensity = 0.05'' would
    set the threshold to 0.05 g for all intensity measure types in the risk 
    calculation.
    If this parameter is not set, the \glsdesc{acr:oqe} extracts the minimum
    thresholds for each intensity measure type from the vulnerability
    models provided, picking the first intensity value for which the mean loss
    ratio is nonzero.

  \item \Verb+return_periods+: this parameter specifies the list of return
    periods (in years) for computing the aggregate loss curve.
    If this parameter is not set, the \glsdesc{acr:oqe} uses a default set of
    return periods for computing the loss curves. The default return periods
    used are from the list: [5, 10, 25, 50, 100, 250, 500, 1000, ...], with 
    its upper bound limited by \Verb+(ses_per_logic_tree_path × investigation_time)+

  \item \Verb+avg_losses+: this boolean parameter specifies whether the average
    asset losses over the time period ``risk\_investigation\_time'' should be
    computed. The default value of this parameter is \Verb+true+.

    \begin{equation*}
    \begin{split}
    average\_loss & = sum(event\_losses) \\
                 & \div (hazard\_investigation\_time \times ses\_per\_logic\_tree\_path) \\
                 & \times risk\_investigation\_time
    \end{split}
    \end{equation*}

\end{itemize}

The above calculation can be run using the command line:

\begin{minted}[fontsize=\footnotesize,frame=single,bgcolor=lightgray]{shell-session}
user@ubuntu:~\$ oq engine --run job.ini
\end{minted}

Computation of the loss tables, loss curves, and average losses for each
individual \gls{asset} in the \gls{exposuremodel} can be resource intensive,
and thus these outputs are not generated by default, unless instructed to by
using the parameters described above.


Starting from \gls{acr:oqe28}, users may also begin a stochastic event based
risk calculation by providing a precomputed set of \glspl{acr:gmf} to the
\gls{acr:oqe}. The following example describes the procedure for this
approach.

\paragraph{Example 2}

This example illustrates a stochastic event based risk calculation which uses
a file listing a precomputed set of \glspl{acr:gmf}. These \glspl{acr:gmf} can
be computed using the \glsdesc{acr:oqe} or some other software. The
\glspl{acr:gmf} must be provided in either the \gls{acr:nrml} schema or the
csv format as presented in Section~\ref{subsec:output_event_based_psha}.
Listing~\ref{lst:output_gmf_xml} shows an example of a \glspl{acr:gmf} file in
the \gls{acr:nrml} schema and Table~\ref{output:gmf_event_based} shows an
example of a \glspl{acr:gmf} file in the csv format. If the \glspl{acr:gmf}
file is provided in the csv format, an additional csv file listing the site
ids must be provided using the parameter \Verb+sites_csv+. See
Table~\ref{output:sitemesh} for an example of the sites csv file, which
provides the association between the site ids in the \glspl{acr:gmf} csv file
with their latitude and longitude coordinates.

Starting from the input \glspl{acr:gmf}, the \gls{acr:oqe} can calculate event
loss tables, loss exceedance curves and probabilistic loss maps for structural
losses, nonstructural losses and occupants. The
job configuration file required for running this stochastic event based risk
calculation starting from a precomputed set of \glspl{acr:gmf} is shown in
Listing~\ref{lst:config_gmf_event_based_risk}.

\begin{listing}[htbp]
  \inputminted[firstline=1,firstnumber=1,fontsize=\scriptsize
  ,frame=single,bgcolor=lightgray,linenos,label=job.ini]{ini}{oqum/risk/verbatim/config_gmf_event_based_risk.ini}
  \caption{Example combined configuration file for running a stochastic event based risk calculation starting from a precomputed set of ground motion fields (\href{https://raw.githubusercontent.com/gem/oq-engine/master/doc/manual/oqum/risk/verbatim/config_gmf_event_based_risk.ini}{Download example})}
  \label{lst:config_gmf_event_based_risk}
\end{listing}
