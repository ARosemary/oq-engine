The seismic source model developed within the Global Seismic Hazard Assessment
Program (GSHAP) is used with the \cite{chiou2008} ground motion prediction
equation to produce the hazard input for this demo. No uncertainties are
considered in the seismic source model and since only one GMPE is being
considered, there will be only one possible path in the logic tree. Therefore,
only one set of seismic hazard curves will be produced. To run the hazard
calculation, the following command needs to be employed:

\begin{minted}[fontsize=\footnotesize,frame=single,bgcolor=lightgray]{shell-session}
user@ubuntu:~\$ oq engine --run job_hazard.ini
\end{minted}

which will produce the following sample hazard output:

\begin{minted}[fontsize=\footnotesize,frame=single,bgcolor=lightgray]{shell-session}
Calculation 8971 completed in 34 seconds. Results:
  id | name
9074 | hcurves
9075 | realizations
\end{minted}

The risk job calculates the probabilistic damage distribution for each asset
in the \gls{exposuremodel} starting from the above generated hazard curves. The
following command launches the risk calculations:

\begin{minted}[fontsize=\footnotesize,frame=single,bgcolor=lightgray]{shell-session}
user@ubuntu:~\$ oq engine --run job_risk.ini --hc 8971
\end{minted}

and the following sample outputs are obtained:

\begin{minted}[fontsize=\footnotesize,frame=single,bgcolor=lightgray]{shell-session}
Calculation 8972 completed in 16 seconds. Results:
  id | name
9076 | damages-rlzs
9077 | damages-stats
\end{minted}
