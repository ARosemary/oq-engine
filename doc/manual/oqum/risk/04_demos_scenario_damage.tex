A rupture of magnitude Mw 7 in the central part of Nepal is considered in this
demo. The characteristics of this rupture (geometry, dip, rake, hypocentre,
upper and lower seismogenic depth) are defined in the \verb+fault_rupture.xml+
file, and the hazard and risk calculation settings are specified in the
\verb+job.ini+ file.

To run the Scenario Damage demo, users should navigate to the folder where the
required files have been placed and employ following command:

\begin{minted}[fontsize=\footnotesize,frame=single,bgcolor=lightgray]{shell-session}
user@ubuntu:~\$ oq engine --run job_hazard.ini,job_risk.ini
\end{minted}

The hazard calculation should produce the following outputs:

\begin{minted}[fontsize=\footnotesize,frame=single,bgcolor=lightgray]{shell-session}
Calculation 8967 completed in 4 seconds. Results:
  id | name
9060 | gmfs
9061 | realizations
\end{minted}

and the following outputs should be produced by the risk calculation:

\begin{minted}[fontsize=\footnotesize,frame=single,bgcolor=lightgray]{shell-session}
Calculation 8968 completed in 16 seconds. Results:
  id | name
9062 | losses_by_asset
9063 | losses_by_taxon
9064 | losses_total
9065 | dmg_by_asset_and_collapse_map
9066 | dmg_by_taxon
9067 | dmg_total
\end{minted}
