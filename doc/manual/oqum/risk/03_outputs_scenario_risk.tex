The Scenario Risk Calculator produces the following set of output files:

\begin{enumerate}

  \item \Verb+agglosses+: this file contains the aggregated scenario
    loss statistics for the entire portfolio of \glspl{asset} defined
    in the \gls{exposuremodel}. The mean (\Verb+mean+) and standard
    deviation (\Verb+stddev+) of the total loss for the portfolio of
    \glspl{asset} are listed in this file.

  \item \Verb+losses_by_asset+: this file contains mean (\Verb+mean+) and
    associated standard deviation (\Verb+stddev+) of the scenario loss for all
    \glspl{asset} at each of the unique locations in the \gls{exposuremodel}.

  \item \Verb+losses_by_event+: this file contains the total loss for the
    portfolio of \glspl{asset} defined in the \gls{exposuremodel} for each
    realization of the scenario generated in the Monte Carlo simulation process.

\end{enumerate}

In addition, if the OpenQuake-QGIS \gls{acr:irmt} plugin is used for
visualizing or exporting the results from a Scenario Risk Calculation, the
following additional outputs can be exported:

\begin{enumerate}
\setcounter{enumi}{3}

  \item \Verb+losses_by_tag+: this file contains the scenario
    loss statistics for each of the \glspl{tag} defined in the
    \gls{exposuremodel}. For each \gls{tag}, the mean (\Verb+mean+)
    and associated standard deviation (\Verb+stddev+)
    of the losses for each tag are listed in this file.

\end{enumerate}

If the calculation involves multiple \glspl{acr:gmpe}, separate output files
are generated for each of the above outputs, for each of the different
\glspl{acr:gmpe} used in the calculation.

These different output files for Scenario Risk calculations are described in
more detail in the following subsections.


\subsection{Scenario loss statistics}
\label{subsec:scenario_loss_statistics}

\subsubsection{Asset loss statistics}
\label{subsubsec:scenario_asset_loss_statistics}

This output is always produced for a Scenario Risk calculation and comprises a
mean total loss and associated standard deviation for each of the individual
\glspl{asset} defined in the \gls{exposuremodel} that fall within the
\Verb+region_constraint+ and have a computed \gls{acr:gmf} value available
within the defined \Verb+asset_hazard_distance+. These results are stored in a
comma separate value (.csv) file as illustrated in the example shown in
Table~\ref{output:scenario_loss_asset}.

\begin{table}[htbp]
\centering
\begin{tabular}{llccrrc}
\hline
\rowcolor{lightgray}
& & & & \textbf{structural} & \textbf{structural} & \ldots \\
\rowcolor{lightgray}
\textbf{asset\_ref} & \textbf{taxonomy} & \textbf{lon} & \textbf{lat} & \textbf{mean} & \textbf{stddev} & \ldots \\ \hline
a3 & wood & -122.57000 & 38.11300 & 686,626 & 1,070,680 & \ldots \\
a2 & concrete & -122.11400 & 38.11300 & 1,496,360 & 2,121,790 & \ldots \\
a5 & wood & -122.00000 & 37.91000 & 3,048,910 & 4,339,480 & \ldots \\
a4 & steel & -122.00000 & 38.00000 & 9,867,070 & 15,969,600 & \ldots \\
a1 & wood & -122.00000 & 38.11300 & 12,993,800 & 22,136,700 & \ldots \\
a6 & concrete & -122.00000 & 38.22500 & 5,632,180 & 9,508,760 & \ldots \\
a7 & wood & -121.88600 & 38.11300 & 2,966,190 & 5,270,480 & \ldots \\
\hline
\end{tabular}
\caption{Example of a scenario asset loss distribution output file}
\label{output:scenario_loss_asset}
\end{table}




\subsubsection{Tag loss statistics}
\label{subsubsec:scenario_tag_loss_statistics}

If the OpenQuake-QGIS \gls{acr:irmt} plugin is used for visualizing or
exporting the results from a Scenario Risk Calculation, the total expected
losses for assets of each \gls{tag} will be computed and made available for
export as a csv file. This distribution of losses per asset \gls{tag} is
depicted in the example output file snippet in
Table~\ref{output:scenario_loss_tag}.

\begin{table}[htbp]
\centering
\begin{tabular}{lrrr}
\hline
\rowcolor{lightgray}
\textbf{tag} & \textbf{contents} & \textbf{nonstructural} & \textbf{structural} \\ \hline
taxonomy=wood & 526,754.0 & 759,653.0 & 393,912.0 \\
taxonomy=concrete & 587,773.0 & 1,074,620.0 & 142,571.0 \\
taxonomy=steel & 407,821.0 & 923,281.0 & 197,341.0 \\
\hline
\end{tabular}
\caption{Example of a scenario loss distribution per tag output file}
\label{output:scenario_loss_tag}
\end{table}



The output file lists the mean loss aggregated for each \glspl{tag}
present in the exposure model and selected by the for all loss types (amongst ``structural'',
``nonstructural'', ``contents'', or ``business\_interruption'') for which a
\gls{vulnerabilitymodel} file was provided in the configuration file.


\subsubsection{Total loss statistics}
\label{subsubsec:scenario_total_loss_statistics}

If the OpenQuake-QGIS \gls{acr:irmt} plugin is used for visualizing or
exporting the results from a Scenario Risk Calculation, the mean total loss
and associated standard deviation for the selected earthquake rupture will be
computed and made available for export as a csv file, as illustrated in the
example shown in Table~\ref{output:scenario_loss_total}.

\begin{table}[htbp]
\centering
\begin{tabular}{llrr}

\hline
\rowcolor{lightgray}
\bf{LossType} & \bf{Unit} & \bf{Mean} & \bf{Standard Deviation} \\
\hline
structural & USD & 8717775315.66 & 2047771108.36 \\
\hline

\end{tabular}
\caption{Example of a scenario total loss output file}
\label{output:scenario_loss_total}
\end{table}


\subsection{Scenario losses by event}
\label{subsec:scenario_losses_event}

The losses by event output lists the total losses for each realization of the
scenario generated in the Monte Carlo simulation process for all loss types
for which a \gls{vulnerabilitymodel} file was provided in the configuration
file. These results are exported in a comma separate value (.csv) file as
illustrated in the example shown in Table~\ref{output:scenario_loss_event}.

\begin{table}[htbp]
\centering
\begin{tabular}{crr}

\hline
\rowcolor{lightgray}
\bf{event} & \bf{structural} & \bf{nonstructural} \\
\hline
1     & 2,194.74   & 20,767.00     \\
2     & 4,037.57   & 20,905.70     \\
3     & 2,950.80   & 18,635.50     \\
4     & 7,787.75   & 19,041.40     \\
5     & 3,964.19   & 30,982.80     \\
6     & 19,394.60  & 40,274.60     \\
⋮     & ⋮          & ⋮             \\
\hline

\end{tabular}
\caption{Example of a scenario losses by event output file}
\label{output:scenario_loss_event}
\end{table}


