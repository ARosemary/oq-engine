The Scenario Risk Calculator produces the following set of output files:

\begin{enumerate}

  \item \Verb+agglosses+: this file contains the aggregated scenario
    loss statistics for the entire portfolio of \glspl{asset} defined
    in the \gls{exposuremodel}. The mean (\Verb+mean+) and standard
    deviation (\Verb+stddev+) of the total loss for the portfolio of
    \glspl{asset} are listed in this file.

  \item \Verb+losses_by_tag+: this file contains the scenario
    loss statistics for each of the \glspl{taxonomy} defined in the
    \gls{exposuremodel}. For each \gls{taxonomy}, the mean (\Verb+mean+)
    and associated standard deviation (\Verb+stddev+)
    of the losses for each taxonomy are listed in this file.

  \item \Verb+losses_by_asset+: this file contains mean (\Verb+mean+) and
    associated standard deviation (\Verb+stddev+) of the scenario loss for all
    \glspl{asset} at each of the unique locations in the \gls{exposuremodel}.

  \item \Verb+losses_by_event+: this file contains the total loss for the
    portfolio of \glspl{asset} defined in the \gls{exposuremodel} for each
    realization of the scenario generated in the Monte Carlo simulation process.


\end{enumerate}

If the calculation involves multiple \glspl{acr:gmpe}, separate output files
are generated for each of the above outputs, for each of the different
\glspl{acr:gmpe} used in the calculation.

These different output files for Scenario Risk calculations are described in
more detail in the following subsections.


\subsection{Scenario loss statistics}
\label{subsec:scenario_loss_statistics}

\subsubsection{Asset loss statistics}
\label{subsubsec:scenario_asset_loss_statistics}

This output is always produced for a Scenario Risk calculation and comprises a
mean total loss and associated standard deviation for each of the individual
\glspl{asset} defined in the \gls{exposuremodel} that fall within the
\Verb+region_constraint+ and have a computed \gls{acr:gmf} value available
within the defined \Verb+asset_hazard_distance+. These results are stored in a
comma separate value (.csv) file as illustrated in the example shown in
Table~\ref{output:scenario_loss_asset}.

\begin{table}[htbp]
\centering
\begin{tabular}{llccrrc}
\hline
\rowcolor{lightgray}
& & & & \textbf{structural} & \textbf{structural} & \ldots \\
\rowcolor{lightgray}
\textbf{asset\_ref} & \textbf{taxonomy} & \textbf{lon} & \textbf{lat} & \textbf{mean} & \textbf{stddev} & \ldots \\ \hline
a3 & wood & -122.57000 & 38.11300 & 686,626 & 1,070,680 & \ldots \\
a2 & concrete & -122.11400 & 38.11300 & 1,496,360 & 2,121,790 & \ldots \\
a5 & wood & -122.00000 & 37.91000 & 3,048,910 & 4,339,480 & \ldots \\
a4 & steel & -122.00000 & 38.00000 & 9,867,070 & 15,969,600 & \ldots \\
a1 & wood & -122.00000 & 38.11300 & 12,993,800 & 22,136,700 & \ldots \\
a6 & concrete & -122.00000 & 38.22500 & 5,632,180 & 9,508,760 & \ldots \\
a7 & wood & -121.88600 & 38.11300 & 2,966,190 & 5,270,480 & \ldots \\
\hline
\end{tabular}
\caption{Example of a scenario asset loss distribution output file}
\label{output:scenario_loss_asset}
\end{table}




\subsubsection{Taxonomy loss statistics}
\label{subsubsec:scenario_taxonomy_loss_statistics}

The Scenario Risk calculator also estimates the total expected losses for
assets of each \gls{taxonomy}. This distribution of losses per building
\gls{taxonomy} is depicted in the example output file snippet in
Table~\ref{output:scenario_loss_taxonomy}.

\begin{table}[htbp]
\centering
\begin{tabular}{lrrr}
\hline
\rowcolor{lightgray}
\textbf{taxonomy} & \textbf{contents} & \textbf{nonstructural} & \textbf{structural} \\ \hline
wood & 526,754.0 & 759,653.0 & 393,912.0 \\
concrete & 587,773.0 & 1,074,620.0 & 142,571.0 \\
steel & 407,821.0 & 923,281.0 & 197,341.0 \\
\hline
\end{tabular}
\caption{Example of a scenario loss distribution per typology output file}
\label{output:scenario_loss_taxonomy}
\end{table}



The output file lists the mean loss aggregated for each building typology
found in the exposure model for all loss types (amongst ``structural'',
``nonstructural'', ``contents'', or ``business\_interruption'') for which a
\gls{vulnerabilitymodel} file was provided in the configuration file.


\subsubsection{Total loss statistics}
\label{subsubsec:scenario_total_loss_statistics}

This output is always produced for a Scenario Risk calculation and comprises a
mean total loss and associated standard deviation for the selected earthquake
rupture. These results are stored in a comma separate value (.csv) file as
illustrated in the example shown in Table~\ref{output:scenario_loss_total}.

\begin{table}[htbp]
\centering
\begin{tabular}{llrr}

\hline
\rowcolor{lightgray}
\bf{LossType} & \bf{Unit} & \bf{Mean} & \bf{Standard Deviation} \\
\hline
structural & USD & 8717775315.66 & 2047771108.36 \\
\hline

\end{tabular}
\caption{Example of a scenario total loss output file}
\label{output:scenario_loss_total}
\end{table}

The important attributes in a scenario total loss statistics output file are
described below:


\begin{itemize}

  \item \Verb+LossType+: the type of losses that are being stored. This
    parameter is taken from the \gls{vulnerabilitymodel} that was used in the
    loss calculations (e.g. fatalities, economic loss).

  \item \Verb+Unit+: this attribute defines the units in which the losses are
    being measured (e.g. USD or EUR). These units are the same as those defined
    in the \gls{exposuremodel} used for the calculation.

  \item \Verb+Mean+: the mean total loss across the portfolio of assets for the
    selected earthquake rupture.

  \item \Verb+Standard Deviation+: the standard deviation of the total loss 
    across the portfolio of assets for the selected earthquake rupture.

\end{itemize} 