The parameter \Verb+calculation_mode+ needs to be set to
\Verb+event_based_damage+ in order to use this calculator.

Most of the job configuration parameters required for running a stochastic
event based damage calculation are the same as those described in the previous
sections for the scenario damage calculator and the classical probabilistic damage
calculator. The remaining parameters specific to the stochastic event based
damage calculator are illustrated through the example below.


\paragraph{Example 1}

This example illustrates a stochastic event based damage calculation which uses
a single configuration file to first compute the \glspl{acr:ses} and
\glspl{acr:gmf} for the given source model and ground motion model, and then
calculate event loss tables, loss exceedance curves and probabilistic
loss maps for structural losses, nonstructural losses and occupants,
based on the \glspl{acr:gmf}. The job configuration file required for
running this stochastic event based damage calculation is shown in
Listing~\ref{lst:config_event_based_damage_combined}.

\begin{listing}[htbp]
  \inputminted[firstline=1,firstnumber=1,fontsize=\scriptsize
  ,frame=single,bgcolor=lightgray,linenos,label=job.ini]{ini}{oqum/risk/verbatim/config_event_based_damage_combined.ini}
  \caption{Example combined configuration file for running a stochastic event based damage calculation (\href{https://raw.githubusercontent.com/gem/oq-engine/master/doc/manual/oqum/risk/verbatim/config_event_based_damage_combined.ini}{Download example})}
  \label{lst:config_event_based_damage_combined}
\end{listing}

Similar to that the procedure described for the Scenario Damage calculator, a
Monte Carlo sampling process is also employed in this calculator to take into
account the uncertainty in the conditional loss ratio at a particular
intensity level. Hence, the parameters \Verb+asset_correlation+ and
\Verb+master_seed+ may be defined as previously described for the Scenario
Damage calculator in Section~\ref{sec:config_scenario_damage}. The parameter
``risk\_investigation\_time'' specifies the time period for which the average
damage values will be calculated, similar to the
Classical Probabilistic Damage calculator. If this parameter is not provided in
the risk job configuration file, the time period used is the same as that
specifed in the hazard calculation using the parameter ``investigation\_time''.

The new parameters introduced in this example are described below:

\begin{itemize}

  \item \Verb+minimum_intensity+: this optional parameter specifies the minimum
    intensity levels for each of the intensity measure types in the risk model.
    Ground motion fields where each ground motion value is less than the 
    specified minimum threshold are discarded. This helps speed up calculations
    and reduce memory consumption by considering only those ground motion fields
    that are likely to contribute to losses. It is also possible to set the same
    threshold value for all intensity measure types by simply providing a single
    value to this parameter. For instance: ``minimum\_intensity = 0.05'' would
    set the threshold to 0.05 g for all intensity measure types in the risk 
    calculation.
    If this parameter is not set, the \glsdesc{acr:oqe} extracts the minimum
    thresholds for each intensity measure type from the vulnerability
    models provided, picking the lowest intensity value for which a mean loss
    ratio is provided.

  \item \Verb+return_periods+: this parameter specifies the list of return
    periods (in years) for computing the asset / aggregate damage curves.
    If this parameter is not set, the \glsdesc{acr:oqe} uses a default set of
    return periods for computing the loss curves. The default return periods
    used are from the list: [5, 10, 25, 50, 100, 250, 500, 1000, ...], with 
    its upper bound limited by \Verb+(ses_per_logic_tree_path × investigation_time)+

    \begin{equation*}
    \begin{split}
    average\_damages & = sum(event\_damages) \\
                 & \div (hazard\_investigation\_time \times ses\_per\_logic\_tree\_path) \\
                 & \times risk\_investigation\_time
    \end{split}
    \end{equation*}

\end{itemize}

The above calculation can be run using the command line:

\begin{minted}[fontsize=\footnotesize,frame=single,bgcolor=lightgray]{shell-session}
user@ubuntu:~\$ oq engine --run job.ini
\end{minted}

Computation of the damage curves, and average damages for each
individual \gls{asset} in the \gls{exposuremodel} can be resource intensive,
and thus these outputs are not generated by default.