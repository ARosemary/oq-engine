This Chapter summarises the structure of the information necessary to define
the different input data to be used with the \glsdesc{acr:oqe} risk
calculators. Input data for scenario-based and probabilistic seismic damage
and risk analysis using the \glsdesc{acr:oqe} are organised into:

\begin{itemize}

  \item An exposure model file in the NRML format, as described in 
    Section~\ref{sec:exposure}.

  \item A file describing the \gls{vulnerabilitymodel}
    (Section~\ref{sec:vulnerability}) for loss calculations, or a 
  	file describing the \gls{fragilitymodel} (Section~\ref{sec:fragility})
    for damage calculations. Optionally, a file describing the
    \gls{consequencemodel} (Section~\ref{sec:consequence}) can also be
  	provided in order to calculate losses from the estimated damage
  	distributions.

  \item A general calculation configuration file.

  \item Hazard inputs. These include hazard curves for the classical
    probabilistic damage and risk calculators, ground motion fields for the
    scenario damage and risk calculators, or stochastic event sets for the
    probabilistic event based calculators. As of \glsdesc{acr:oqe21}, in
    general, there are five different ways in which hazard calculation
    parameters or results can be provided to the \glsdesc{acr:oqe} in order to
    run the subsequent risk calculations:

    \begin{itemize}

      \item Use a single configuration file for running the hazard and risk
      calculations sequentially (preferred)

      \item Use separate configuration files for running the hazard and risk
      calculations sequentially (legacy)

      \item Use a configuration file for the risk calculation along with all
      hazard outputs from a previously completed, compatible
      \glsdesc{acr:oqe} hazard calculation

      % \item Use a configuration file for the risk calculation along with a
      % specific hazard output from a previously completed, compatible
      % \glsdesc{acr:oqe} hazard calculation

      \item Use a configuration file for the risk calculation along with
      hazard input files in the OpenQuake NRML format

    \end{itemize}

\end{itemize}

The file formats for \glspl{exposuremodel}, \glspl{fragilitymodel},
\glspl{consequencemodel}, and \glspl{vulnerabilitymodel} have been described
earlier in Chapter~\ref{chap:riskinputs}. The configuration file is the primary
file that provides the \glsdesc{acr:oqe} information regarding both the
definition of the input models (e.g. exposure, site parameters, fragility,
consequence, or vulnerability models) as well as the parameters governing the
risk calculation.

Information regarding the configuration file for running hazard calculations
using the \glsdesc{acr:oqe} can be found in
Section~\ref{sec:hazard_configuration_file}. Some initial mandatory parameters
of the configuration file common to all of the risk calculators are presented
in Listing~\ref{lst:config_example}. The remaining parameters that are
specific to each risk calculator are discussed in subsequent sections.

\begin{listing}[htbp]
  \inputminted[firstline=1,firstnumber=1,fontsize=\footnotesize,frame=single,linenos,bgcolor=lightgray]{ini}{oqum/risk/verbatim/config_example.ini}
  \caption{Example minimal risk calculation configuration file (\href{https://raw.githubusercontent.com/gem/oq-engine/master/doc/manual/oqum/risk/verbatim/config_example.xml}{Download example})}
  \label{lst:config_example}
\end{listing}

\begin{itemize}

  \item \Verb+description+: a parameter that can be used to include some
  information about the type of calculations that are going to be performed.

  \item \Verb+calculation_mode+: this parameter specifies the type of
  calculation to be run. Valid options for the \Verb+calculation_mode+ for
  the risk calculators are: \Verb+scenario_damage+, \Verb+scenario_risk+,
  \Verb+classical_damage+, \Verb+classical_risk+, \Verb+event_based_risk+,
  and \Verb+classical_bcr+.

  \item \Verb+exposure_file+: this parameter is used to specify the path to
  the \gls{exposuremodel} file.

\end{itemize}

Depending on the type of risk calculation, other parameters besides the
aforementioned ones may need to be provided. We illustrate in the following
sections different examples of the configuration file for the different risk
calculators.


\section{Scenario Damage Calculator}
\label{sec:config_scenario_damage}
For this calculator, the parameter \Verb+calculation_mode+ should be set to
\Verb+scenario_damage+.

\paragraph{Example 1}

This example illustrates a scenario damage calculation which uses a single
configuration file to first compute the ground motion fields for the given
rupture model and then calculate damage distribution statistics based on the
ground motion fields. A minimal job configuration file required for running a
scenario damage calculation is shown in
Listing~\ref{lst:config_scenario_damage_combined}.

\begin{listing}[htbp]
  \inputminted[firstline=1,firstnumber=1,fontsize=\footnotesize,frame=single,linenos,bgcolor=lightgray,label=job.ini]{ini}{oqum/risk/verbatim/config_scenario_damage_combined.ini}
  \caption{Example combined configuration file for running a scenario damage calculation (\href{https://raw.githubusercontent.com/gem/oq-engine/master/doc/manual/oqum/risk/verbatim/config_scenario_damage_combined.ini}{Download example})}
  \label{lst:config_scenario_damage_combined}
\end{listing}

The general parameters \Verb+description+ and \Verb+calculation_mode+, and
\Verb+exposure_file+ have already been described earlier. The other parameters
seen in the above example configuration file are described below:

\begin{itemize}

  \item \Verb+rupture_model_file+: a parameter used to define the path
	to the earthquake \gls{rupturemodel} file describing the scenario event.

  \item \Verb+rupture_mesh_spacing+: a parameter used to specify the mesh size
  	(in km) used by the \glsdesc{acr:oqe} to discretize the rupture.
  	Note that the smaller the mesh spacing, the greater will be
  	(1) the precision in the calculation and
  	(2) the computational demand.

  \item \Verb+structural_fragility_file+: a parameter used to define the path
	to the structural \gls{fragilitymodel} file.

\end{itemize}

In this case, the ground motion fields will be computed at each of the
locations of the assets in the exposure model. Ground motion fields will be
generated for each of the intensity measure types found in the provided set of
fragility models. The above calculation can be run using the command line:

\begin{minted}[fontsize=\footnotesize,frame=single,bgcolor=lightgray]{shell-session}
user@ubuntu:~\$ oq engine --run job.ini
\end{minted}

After the calculation is completed, a message similar to the following will be
displayed:

\begin{minted}[fontsize=\footnotesize,frame=single,bgcolor=lightgray]{shell-session}
Calculation 2680 completed in 13 seconds. Results:
  id | name
5069 | Average Asset Damages
\end{minted}

Note that one or more of the following parameters can be used in the same job
configuration file to provide the corresponding fragility model files:

\begin{itemize}

  \item \Verb+structural_fragility_file+: a parameter used to define the path
    to a structural \gls{fragilitymodel} file

  \item \Verb+nonstructural_fragility_file+: a parameter used to define the path
    to a nonstructural \gls{fragilitymodel} file

  \item \Verb+contents_fragility_file+: a parameter used to define the path
    to a contents \gls{fragilitymodel} file

  \item \Verb+business_interruption_fragility_file+: a parameter used to define
    the path to a business interruption \gls{fragilitymodel} file

\end{itemize}

It is important that the \Verb+lossCategory+ parameter in the metadata section
for each provided fragility model file (``structural'', ``nonstructural'',
``contents'', or ``business\_interruption'') should match the loss type
defined in the configuration file by the relevant keyword above.


\paragraph{Example 2}

This example illustrates a scenario damage calculation which uses separate
configuration files for the hazard and risk parts of a scenario damage
assessment. The first configuration file shown in
Listing~\ref{lst:config_scenario_damage_hazard} contains input models and
parameters required for the computation of the ground motion fields due to a
given rupture. The second configuration file shown in
Listing~\ref{lst:config_scenario_damage} contains input models and parameters
required for the calculation of the damage distribution for a portfolio of
assets due to the ground motion fields.

\begin{listing}[htbp]
  \inputminted[firstline=1,firstnumber=1,fontsize=\footnotesize,frame=single,linenos,bgcolor=lightgray,label=job\_hazard.ini]{ini}{oqum/risk/verbatim/config_scenario_hazard.ini}
  \caption{Example hazard configuration file for a scenario damage calculation (\href{https://raw.githubusercontent.com/gem/oq-engine/master/doc/manual/oqum/risk/verbatim/config_scenario_hazard.ini}{Download example})}
  \label{lst:config_scenario_damage_hazard}
\end{listing}

\begin{listing}[htbp]
  \inputminted[firstline=1,firstnumber=1,fontsize=\footnotesize,frame=single,linenos,bgcolor=lightgray,label=job\_damage.ini]{ini}{oqum/risk/verbatim/config_scenario_damage.ini}
  \caption{Example risk configuration file for a scenario damage calculation (\href{https://raw.githubusercontent.com/gem/oq-engine/master/doc/manual/oqum/risk/verbatim/config_scenario_damage.ini}{Download example})}
  \label{lst:config_scenario_damage}
\end{listing}


In this example, the set of intensity measure types for which the ground
motion fields should be generated is specified explicitly in the configuration
file using the parameter \Verb+intensity_measure_types+. If the hazard
calculation outputs are intended to be used as inputs for a subsequent
scenario damage or risk calculation, the set of intensity measure types
specified here must include all intensity measure types that are used in the
fragility or vulnerability models for the subsequent damage or risk
calculation.

In the hazard configuration file illustrated above
(Listing~\ref{lst:config_scenario_damage_hazard}), the list of sites at which
the ground motion values will be computed is provided in a CSV file, specified
using the \Verb+sites_csv+ parameter. The sites used for the hazard
calculation need not be the same as the locations of the assets in the
exposure model used for the following risk calculation. In such cases, it is
recommended to set a reasonable search radius (in km) using the
\Verb+asset_hazard_distance+ parameter for the \glsdesc{acr:oqe} to look for
available hazard values, as shown in the job\_damage.ini example file above.

The only new parameters introduced in risk configuration file for this example
(Listing~\ref{lst:config_scenario_damage}) are the \Verb+region_constraint+,
\Verb+asset_hazard_distance+, and \Verb+time_event+ parameters, which are
described below; all other parameters have already been described in earlier
examples.

\begin{itemize}

  \item \Verb+region_constraint+: this is an optional parameter, applicable
    only to risk calculations, which defines the polygon that will be used for
    filtering the assets from the exposure model. Assets outside of this region
    will not be considered in the risk calculations. This region is defined
    using pairs of coordinates that indicate the vertices of the polygon, which
    should be listed in the Well-known text (WKT) format:

    region\_constraint = lon\_1 lat\_1, lon\_2 lat\_2, ..., lon\_n lat\_n

    For each point, the longitude is listed first, followed by the latitude,
    both in decimal degrees. The list of points defining the polygon can be
    provided either in a clockwise or counter-clockwise direction.

    If the \Verb+region_constraint+ is not provided, all assets in the exposure
    model are considered for the risk calculation.

    This parameter is useful in cases where the exposure model covers a region
    larger than the one that is of interest in the current calculation.

  \item \Verb+asset_hazard_distance+: this parameter indicates the maximum
    allowable distance between an \gls{asset} and the closest hazard input.
    Hazard inputs can include hazard curves or ground motion intensity values.
    If no hazard input site is found within the radius defined by the
    \Verb+asset_hazard_distance+, the asset is skipped and a message is
    provided mentioning the id of the asset that is affected by this issue.

    If multiple hazard input sites are found within the radius defined by the
    this parameter, the hazard input site with the shortest distance from the
    asset location is associated with the asset. It is possible that the
    associated hazard input site might be located outside the polygon defined
    by the \Verb+region_constraint+.

  \item \Verb+time_event+: this parameter indicates the time of day at which
    the event occurs. The values that this parameter can be set to are 
    currently limited to one of the three strings: \Verb+day+, \Verb+night+,
    and \Verb+transit+. This parameter will be used to compute the number of
    fatalities based on the number of occupants present in the various
    \glspl{asset} at that time of day, as specified in the exposure model.

\end{itemize}


Now, the above calculations described by the two configuration files
``job\_hazard.ini'' and ``job\_damage.ini'' can be run separately. The
calculation id for the hazard calculation should be
provided to the \glsdesc{acr:oqe} while running the risk calculation using the
option \Verb+--hazard-calculation-id+ (or \Verb+--hc+). This is shown below:

\begin{minted}[fontsize=\footnotesize,frame=single,bgcolor=lightgray]{shell-session}
user@ubuntu:~\$ oq engine --run job_hazard.ini
\end{minted}

After the hazard calculation is completed, a message similar to the one below
will be displayed in the terminal:

\begin{minted}[fontsize=\footnotesize,frame=single,bgcolor=lightgray]{shell-session}
Calculation 2681 completed in 4 seconds. Results:
  id | name
5072 | Ground Motion Fields
\end{minted}

In the example above, the calculation~id of the hazard calculation is 2681.
There is only one output from this calculation, i.e., the \glspl{acr:gmf}.

The risk calculation for computing the damage distribution statistics for the
portfolio of \glspl{asset} can now be run using:

\begin{minted}[fontsize=\footnotesize,frame=single,bgcolor=lightgray]{shell-session}
user@ubuntu:~\$ oq engine --run job_damage.ini --hc 2681
\end{minted}

After the calculation is completed, a message similar to the one listed above
in Example~1 will be displayed.

In order to retrieve the calculation~id of a previously run hazard calculation,
the option \Verb+--list-hazard-calculations+ (or \Verb+--lhc+) can be used to
display a list of all previously run hazard calculations:

\begin{minted}[fontsize=\footnotesize,frame=single,bgcolor=lightgray]{shell-session}
job_id |     status |         start_time |         description
  2609 | successful | 2015-12-01 14:14:14 | Mid Nepal earthquake
  ...
  2681 | successful | 2015-12-12 10:00:00 | Scenario hazard example
\end{minted}

The option \Verb+--list-outputs+ (or \Verb+--lo+) can be used to display a
list of all outputs generated during a particular calculation. For instance,

\begin{minted}[fontsize=\footnotesize,frame=single,bgcolor=lightgray]{shell-session}
user@ubuntu:~\$ oq engine --lo 2681
\end{minted}

will produce the following display:

\begin{minted}[fontsize=\footnotesize,frame=single,bgcolor=lightgray]{shell-session}
  id | name
5072 | Ground Motion Fields
\end{minted}


\paragraph{Example 3}

The example shown in Listing~\ref{lst:config_scenario_damage_gmf_xml} illustrates
a scenario damage calculation which uses a file listing a precomputed set of
\glspl{acr:gmf}. These \glspl{acr:gmf} can be computed using the
\glsdesc{acr:oqe} or some other software. The \glspl{acr:gmf} must be provided
in either the \gls{acr:nrml} schema or the csv format as presented in
Section~\ref{subsec:output_scenario_hazard}. The damage distribution is
computed based on the provided \glspl{acr:gmf}.
Listing~\ref{lst:output_gmf_scenario_xml} shows an example of a
\glspl{acr:gmf} file in the \gls{acr:nrml} schema and
Table~\ref{output:gmf_scenario} shows an example of a \glspl{acr:gmf} file in
the csv format. If the \glspl{acr:gmf} file is provided in the csv format, an
additional csv file listing the site ids must be provided using the parameter
\Verb+sites_csv+. See Table~\ref{output:sitemesh} for an example of the sites
csv file, which provides the association between the site ids in the
\glspl{acr:gmf} csv file with their latitude and longitude coordinates.

\begin{listing}[htbp]
  \inputminted[firstline=1,firstnumber=1,fontsize=\footnotesize,frame=single,linenos,bgcolor=lightgray,label=job.ini]{ini}{oqum/risk/verbatim/config_scenario_damage_gmf_xml.ini}
  \caption{Example configuration file for a scenario damage calculation using a precomputed set of ground motion fields (\href{https://raw.githubusercontent.com/gem/oq-engine/master/doc/manual/oqum/risk/verbatim/config_scenario_damage_gmf_xml.ini}{Download example})}
  \label{lst:config_scenario_damage_gmf_xml}
\end{listing}

\begin{itemize}

  \item \Verb+gmfs_file+: a parameter used to define the path
    to the \glspl{acr:gmf} file in the \gls{acr:nrml} schema. This file must
    define \glspl{acr:gmf} for all of the intensity measure types used in the
    \gls{fragilitymodel}.

\end{itemize}

\begin{listing}[htbp]
  \inputminted[firstline=1,firstnumber=1,fontsize=\footnotesize,frame=single,linenos,bgcolor=lightgray,label=job.ini]{ini}{oqum/risk/verbatim/config_scenario_damage_gmf_csv.ini}
  \caption{Example configuration file for a scenario damage calculation using a precomputed set of ground motion fields (\href{https://raw.githubusercontent.com/gem/oq-engine/master/doc/manual/oqum/risk/verbatim/config_scenario_damage_gmf_csv.ini}{Download example})}
  \label{lst:config_scenario_damage_gmf_csv}
\end{listing}

\begin{itemize}

  \item \Verb+gmfs_csv+: a parameter used to define the path
    to the \glspl{acr:gmf} file in the csv format. This file must
    define \glspl{acr:gmf} for all of the intensity measure types used in the
    \gls{fragilitymodel}.
    (\href{https://raw.githubusercontent.com/gem/oq-engine/master/doc/manual/oqum/risk/verbatim/input_scenario_gmfs.csv}{Download an example file here}).

  \item \Verb+sites_csv+: a parameter used to define the path
    to the sites file in the csv format. This file must
    define site id, longitude, and latitude for all of the sites for the
    \glspl{acr:gmf} file provided using the \Verb+gmfs_csv+ parameter. 
    (\href{https://raw.githubusercontent.com/gem/oq-engine/master/doc/manual/oqum/risk/verbatim/input_scenario_sites.csv}{Download an example file here}).

\end{itemize}

The above calculation(s) can be run using the command line:

\begin{minted}[fontsize=\footnotesize,frame=single,bgcolor=lightgray]{shell-session}
user@ubuntu:~\$ oq engine --run job.ini
\end{minted}


\paragraph{Example 4}

This example illustrates a the hazard job configuration file for a scenario
damage calculation which uses two \glspl{acr:gmpe} instead of only one.
Currently, the set of \glspl{acr:gmpe} to be used for a scenario calculation
can be specified using a logic tree file, as demonstrated in
\ref{subsec:gmlt}. As of \glsdesc{acr:oqe18}, the weights in the logic tree
are ignored, and a set of \glspl{acr:gmf} will be generated for each
\gls{acr:gmpe} in the logic tree file. Correspondingly, damage distribution
statistics will be generated for each set of \gls{acr:gmf}.

The file shown in Listing~\ref{lst:input_scenario_gmlt} lists the two
\glspl{acr:gmpe} to be used for the hazard calculation:

\begin{listing}[htbp]
  \inputminted[firstline=1,firstnumber=1,fontsize=\footnotesize,frame=single,linenos,bgcolor=lightgray,label=gsim\_logic\_tree.xml]{xml}{oqum/risk/verbatim/input_scenario_gmlt.xml}
  \caption{Example ground motion logic tree for a scenario calculation (\href{https://raw.githubusercontent.com/gem/oq-engine/master/doc/manual/oqum/risk/verbatim/input_scenario_gmlt.xml}{Download example})}
  \label{lst:input_scenario_gmlt}
\end{listing}

The only change that needs to be made in the hazard job configuration file is
to replace the \Verb+gsim+ parameter with \Verb+gsim_logic_tree_file+, as
demonstrated in Listing~\ref{lst:config_scenario_hazard_gmlt}.

\begin{listing}[htbp]
  \inputminted[firstline=1,firstnumber=1,fontsize=\footnotesize,frame=single,linenos,bgcolor=lightgray,label=job\_hazard.ini]{ini}{oqum/risk/verbatim/config_scenario_hazard_gmlt.ini}
  \caption{Example configuration file for a scenario damage calculation using a logic-tree file (\href{https://raw.githubusercontent.com/gem/oq-engine/master/doc/manual/oqum/risk/verbatim/config_scenario_hazard_gmlt.ini}{Download example})}
  \label{lst:config_scenario_hazard_gmlt}
\end{listing}


\paragraph{Example 5}

This example illustrates a scenario damage calculation which specifies
fragility models for calculating damage to structural and nonstructural
components of structures, and also specifies \gls{consequencemodel} files for
calculation of the corresponding losses.

A minimal job configuration file required for running a scenario damage
calculation followed by a consequences analysis is shown in
Listing~\ref{lst:config_scenario_damage_consequences}.

\begin{listing}[htbp]
  \inputminted[firstline=1,firstnumber=1,fontsize=\footnotesize,frame=single,linenos,bgcolor=lightgray,label=job.ini]{ini}{oqum/risk/verbatim/config_scenario_damage_consequences.ini}
  \caption{Example configuration file for a scenario damage calculation followed by a consequences analysis (\href{https://raw.githubusercontent.com/gem/oq-engine/master/doc/manual/oqum/risk/verbatim/config_scenario_damage_consequences.ini}{Download example})}
  \label{lst:config_scenario_damage_consequences}
\end{listing}

Note that one or more of the following parameters can be used in the same job
configuration file to provide the corresponding \gls{consequencemodel} files:

\begin{itemize}

  \item \Verb+structural_consequence_file+: a parameter used to define the path
    to a structural \gls{consequencemodel} file

  \item \Verb+nonstructural_consequence_file+: a parameter used to define the path
    to a nonstructural \gls{consequencemodel} file

  \item \Verb+contents_consequence_file+: a parameter used to define the path
    to a contents \gls{consequencemodel} file

  \item \Verb+business_interruption_consequence_file+: a parameter used to define
    the path to a business interruption \gls{consequencemodel} file

\end{itemize}

It is important that the \Verb+lossCategory+ parameter in the metadata section
for each provided \gls{consequencemodel} file (``structural'', ``nonstructural'',
``contents'', or ``business\_interruption'') should match the loss type
defined in the configuration file by the relevant keyword above.

The above calculation can be run using the command line:

\begin{minted}[fontsize=\footnotesize,frame=single,bgcolor=lightgray]{shell-session}
user@ubuntu:~\$ oq engine --run job.ini
\end{minted}

After the calculation is completed, a message similar to the following will be
displayed:

\begin{minted}[fontsize=\footnotesize,frame=single,bgcolor=lightgray]{shell-session}
Calculation 1579 completed in 37 seconds. Results:
  id | name
8990 | Average Asset Losses
8993 | Average Asset Damages
\end{minted}


\section{Scenario Risk Calculator}
\label{sec:config_scenario_risk}
In order to run this calculator, the parameter \Verb+calculation_mode+ needs
to be set to \Verb+scenario_risk+. 

Most of the job configuration parameters required for running a scenario risk
calculation are the same as those described in the previous section for the
scenario damage calculator. The remaining parameters specific to the scenario
risk calculator are illustrated through the examples below.


\paragraph{Example 1}

This example illustrates a scenario risk calculation which uses a single
configuration file to first compute the ground motion fields for the given
rupture model and then calculate loss statistics for structural losses,
nonstructural losses, and insured structural losses, based on the ground
motion fields. The job configuration file required for running this scenario
risk calculation is shown in Listing~\ref{lst:config_scenario_risk_combined}.

\begin{listing}[htbp]
  \inputminted[firstline=1,firstnumber=1,fontsize=\footnotesize,frame=single,linenos,bgcolor=lightgray,label=job.ini]{ini}{oqum/risk/verbatim/config_scenario_risk_combined.ini}
  \caption{Example combined configuration file for a scenario risk calculation (\href{https://raw.githubusercontent.com/GEMScienceTools/oq-engine-docs/master/oqum/risk/verbatim/config_scenario_risk_combined.ini}{Download example})}
  \label{lst:config_scenario_risk_combined}
\end{listing}

Whereas a scenario damage calculation requires one or more fragility and/or
consequence models, a scenario risk calculation requires the user to specify
one or more vulnerability model files. Note that one or more of the following
parameters can be used in the same job configuration file to provide the
corresponding vulnerability model files:

\begin{itemize}

  \item \Verb+structural_vulnerability_file+: this parameter is used to
    specify the path to the structural \gls{vulnerabilitymodel} file

  \item \Verb+nonstructural_vulnerability_file+: this parameter is used to
    specify the path to the nonstructural\gls{vulnerabilitymodel} file

  \item \Verb+contents_vulnerability_file+: this parameter is used to
    specify the path to the contents \gls{vulnerabilitymodel} file

  \item \Verb+business_interruption_vulnerability_file+: this parameter is
    used to specify the path to the business interruption
    \gls{vulnerabilitymodel} file

  \item \Verb+occupants_vulnerability_file+: this parameter is used to
    specify the path to the occupants \gls{vulnerabilitymodel} file

\end{itemize}

It is important that the \Verb+lossCategory+ parameter in the metadata section
for each provided vulnerability model file (``structural'', ``nonstructural'',
``contents'', ``business\_interruption'', or ``occupants'') should match the
loss type defined in the configuration file by the relevant keyword above.

The remaining new parameters introduced in this example are the following:

\begin{itemize}

  \item \Verb+master_seed+: this parameter is used to control the random
    number generator in the loss ratio sampling process. If the same
    \Verb+master_seed+ is defined at each calculation run, the same random loss
    ratios will be generated, thus allowing reproducibility of the results.

  \item \Verb+asset_correlation+: if the uncertainty in the loss ratios
    has been defined within the \gls{vulnerabilitymodel}, users can specify
    a coefficient of correlation that will be used in the Monte Carlo sampling
    process of the loss ratios, between the assets that share the same
    \gls{taxonomy}. If the \Verb+asset_correlation+ is set to one,
    the loss ratio residuals will be perfectly correlated. On the other hand,
    if this parameter is set to zero, the loss ratios will be sampled
    independently. If this parameter is not defined, the
    \glsdesc{acr:oqe} will assume zero correlation in the vulnerability. As of
    \glsdesc{acr:oqe18}, \Verb+asset_correlation+ applies only to continuous
    \glspl{vulnerabilityfunction} using the lognormal or Beta distribution; 
    it does not apply to \glspl{vulnerabilityfunction} defined using the PMF
    distribution. Although partial correlation was supported in previous
    versions of the engine, beginning from \glsdesc{acr:oqe22}, values between
    zero and one are no longer supported due to performance considerations. The
    only two values permitted are \Verb+asset_correlation = 0+ and 
    \Verb+asset_correlation = 1+.

  \item \Verb+insured_losses+: this parameter specifies whether insured losses
    should be calculated; the default value of this parameter is \Verb+false+.
    In order for the \glsdesc{acr:oqe} to be able to compute insured losses, the
    insurance limits and deductibles must be listed for each asset in the 
    exposure model, as described in Example~5 in Section~\ref{sec:exposure}.

  \item \Verb+asset_loss_table+: this parameter
    specifies whether the individual asset and portfolio losses should be
    stored for each realization; the default value of this parameter is
    \Verb+false+ and only mean and standard deviation of the portfolio losses
    across all realizations are stored. If this flag is set to \Verb+true+, a
    matrix containing all of the losses is saved in the datastore and can be
    exported in the .npz format.

\end{itemize}

In this case, the ground motion fields will be computed at each of the
locations of the assets in the exposure model and for each of the intensity
measure types found in the provided set of vulnerability models. The above
calculation can be run using the command line:

\begin{minted}[fontsize=\footnotesize,frame=single,bgcolor=lightgray]{shell-session}
user@ubuntu:~\$ oq engine --run job.ini
\end{minted}

After the calculation is completed, a message similar to the following will be
displayed:

\begin{minted}[fontsize=\footnotesize,frame=single,bgcolor=lightgray]{shell-session}
Calculation 2735 completed in 10 seconds. Results:
  id | name
5328 | agglosses-rlzs
5329 | losses_by_asset
\end{minted}

All of the different ways of running a scenario damage calculation as
illustrated through the examples of the previous section are also applicable
to the scenario risk calculator, though the examples are not repeated here.


\section{Classical Probabilistic Seismic Damage Calculator}
\label{sec:config_classical_damage}
In order to run this calculator, the parameter \Verb+calculation_mode+ needs
to be set to \Verb+classical_damage+.

Most of the job configuration parameters required for running a classical
probabilistic damage calculation are the same as those described in the
section for the scenario damage calculator. The remaining parameters specific
to the classical probabilistic damage calculator are illustrated through the
examples below.

\paragraph{Example 1}

This example illustrates a classical probabilistic damage calculation which
uses a single configuration file to first compute the hazard curves for the
given source model and ground motion model and then calculate damage
distribution statistics based on the hazard curves. A minimal job
configuration file required for running a classical probabilistic damage
calculation is shown in Listing~\ref{lst:config_classical_damage_combined}.

\begin{listing}[htbp]
  \inputminted[firstline=1,firstnumber=1,fontsize=\footnotesize,frame=single,linenos,bgcolor=lightgray,label=job.ini]{ini}{oqum/risk/verbatim/config_classical_damage_combined.ini}
  \caption{Example combined configuration file for a classical probabilistic damage calculation (\href{https://raw.githubusercontent.com/gem/oq-engine/master/doc/manual/oqum/risk/verbatim/config_classical_damage_combined.ini}{Download example})}
  \label{lst:config_classical_damage_combined}
\end{listing}

The general parameters \Verb+description+ and \Verb+calculation_mode+, and
\Verb+exposure_file+ have already been described earlier in
Section~\ref{sec:config_scenario_damage}. The parameters related to the
hazard curves computation have been described earlier in
Section~\ref{subsec:config_classical_psha}.

In this case, the hazard curves will be computed at each of the locations of
the assets in the exposure model, for each of the intensity measure types
found in the provided set of \glspl{fragilitymodel}. The above calculation can be
run using the command line:

\begin{minted}[fontsize=\footnotesize,frame=single,bgcolor=lightgray]{shell-session}
user@ubuntu:~\$ oq engine --run job.ini
\end{minted}

After the calculation is completed, a message similar to the following will be
displayed:

\begin{minted}[fontsize=\footnotesize,frame=single,bgcolor=lightgray]{shell-session}
Calculation 2741 completed in 12 seconds. Results:
  id | name
5359 | Asset Damage Distribution
\end{minted}


\paragraph{Example 2}

This example illustrates a classical probabilistic damage calculation which
uses separate configuration files for the hazard and risk parts of a classical
probabilistic damage assessment. The first configuration file shown in
Listing~\ref{lst:config_classical_damage_hazard} contains input models and parameters
required for the computation of the hazard curves. The second configuration
file shown in Listing~\ref{lst:config_classical_damage} contains input models
and parameters required for the calculation of the probabilistic damage
distribution for a portfolio of assets based on the hazard curves and
fragility models.

\begin{listing}[htbp]
  \inputminted[firstline=1,firstnumber=1,fontsize=\footnotesize,frame=single,linenos,bgcolor=lightgray,label=job\_hazard.ini]{ini}{oqum/risk/verbatim/config_classical_hazard.ini}
  \caption{Example hazard configuration file for a classical probabilistic damage calculation (\href{https://raw.githubusercontent.com/gem/oq-engine/master/doc/manual/oqum/risk/verbatim/config_classical_hazard.ini}{Download example})}
  \label{lst:config_classical_damage_hazard}
\end{listing}

\begin{listing}[htbp]
  \inputminted[firstline=1,firstnumber=1,fontsize=\footnotesize,frame=single,linenos,bgcolor=lightgray,label=job\_damage.ini]{ini}{oqum/risk/verbatim/config_classical_damage.ini}
  \caption{Example risk configuration file for a classical probabilistic damage calculation (\href{https://raw.githubusercontent.com/gem/oq-engine/master/doc/manual/oqum/risk/verbatim/config_classical_damage.ini}{Download example})}
  \label{lst:config_classical_damage}
\end{listing}

Now, the above calculations described by the two configuration files
``job\_hazard.ini'' and ``job\_damage.ini'' can be run sequentially or
separately, as illustrated in Example~2 in
Section~\ref{sec:config_scenario_damage}. The new parameters introduced in the
above example configuration file are described below:

\begin{itemize}

  \item \Verb+risk_investigation_time+: an optional parameter that can be used
    in probabilistic damage or risk calculations where the period of interest
    for the risk calculation is different from the period of interest for the 
    hazard calculation. If this parameter is not explicitly set, the 
    \glsdesc{acr:oqe} will assume that the risk calculation is over the same 
    time period as the preceding hazard calculation.

  \item \Verb+steps_per_interval+: an optional parameter that can be used to
    specify whether discrete fragility functions in the fragility models should
    be discretized further, and if so, how many intermediate steps to use for
    the discretization. Setting 

    steps\_per\_interval = n

    will result in the \glsdesc{acr:oqe} discretizing the discrete fragility
    models using (n - 1) linear interpolation steps between each pair of 
    {intensity level, poe} points.

    The default value of this parameter is one, implying no interpolation.

\end{itemize}


\section{Classical Probabilistic Seismic Risk Calculator}
\label{sec:config_classical_risk}
In order to run this calculator, the parameter \Verb+calculation_mode+ needs
to be set to \Verb+classical_risk+.

Most of the job configuration parameters required for running a classical
probabilistic risk calculation are the same as those described in the previous
section for the classical probabilistic damage calculator. The remaining
parameters specific to the classical probabilistic risk calculator are
illustrated through the examples below.

\paragraph{Example 1}

This example illustrates a classical probabilistic risk calculation which uses
a single configuration file to first compute the hazard curves for the given
source model and ground motion model and then calculate loss exceedance curves
based on the hazard curves. An example job configuration file for running a
classical probabilistic risk calculation is shown in
Listing~\ref{lst:config_classical_risk_combined}.

\begin{listing}[htbp]
  \inputminted[firstline=1,firstnumber=1,fontsize=\footnotesize,frame=single,linenos,bgcolor=lightgray,label=job.ini]{ini}{oqum/risk/verbatim/config_classical_risk_combined.ini}
  \caption{Example combined configuration file for a classical probabilistic risk calculation (\href{https://raw.githubusercontent.com/GEMScienceTools/oq-engine-docs/master/oqum/risk/verbatim/config_classical_risk_combined.ini}{Download example})}
  \label{lst:config_classical_risk_combined}
\end{listing}

Apart from the calculation mode, the only difference with the example job
configuration file shown in Example~1 of
Section~\ref{sec:config_classical_damage} is the use of a vulnerability model
instead of a fragility model.

As with the Scenario Risk calculator, it is possible to specify one or more
\gls{vulnerabilitymodel} files in the same job configuration file, using the
parameters:

\begin{itemize}

  \item \Verb+structural_vulnerability_file+,

  \item \Verb+nonstructural_vulnerability_file+,

  \item \Verb+contents_vulnerability_file+,

  \item \Verb+business_interruption_vulnerability_file+, and/or

  \item \Verb+occupants_vulnerability_file+

\end{itemize}

It is important that the
\Verb+lossCategory+ parameter in the metadata section for each provided
vulnerability model file (``structural'', ``nonstructural'', ``contents'',
``business\_interruption'', or ``occupants'') should match the loss type
defined in the configuration file by the relevant keyword above.

In this case, the hazard curves will be computed at each of the locations of
the \glspl{asset} in the \gls{exposuremodel}, for each of the intensity
measure types found in the provided set of \glspl{vulnerabilitymodel}. The
above calculation can be run using the command line:

\begin{minted}[fontsize=\footnotesize,frame=single,bgcolor=lightgray]{shell-session}
user@ubuntu:~\$ oq engine --run job.ini
\end{minted}

After the calculation is completed, a message similar to the following will be
displayed:

\begin{minted}[fontsize=\footnotesize,frame=single,bgcolor=lightgray]{shell-session}
Calculation 2749 completed in 24 seconds. Results:
  id | name
3980 | loss_curves-stats
3981 | loss_maps-stats
3983 | avg_losses-stats
\end{minted}


\paragraph{Example 2}

This example illustrates a classical probabilistic risk calculation which uses
separate configuration files for the hazard and risk parts of a classical
probabilistic risk assessment. The first configuration file shown in
Listing~\ref{lst:config_classical_risk_hazard} contains input models and
parameters required for the computation of the hazard curves. The second
configuration file shown in Listing~\ref{lst:config_classical_risk} contains
input models and parameters required for the calculation of the loss
exceedance curves and probabilistic loss maps for a portfolio of \glspl{asset}
based on the hazard curves and \glspl{vulnerabilitymodel}.

\begin{listing}[htbp]
  \inputminted[firstline=1,firstnumber=1,fontsize=\footnotesize,frame=single,linenos,bgcolor=lightgray,label=job\_hazard.ini]{ini}{oqum/risk/verbatim/config_classical_hazard.ini}
  \caption{Example hazard configuration file for a classical probabilistic risk calculation (\href{https://raw.githubusercontent.com/GEMScienceTools/oq-engine-docs/master/oqum/risk/verbatim/config_classical_hazard.ini}{Download example})}
  \label{lst:config_classical_risk_hazard}
\end{listing}

\begin{listing}[htbp]
  \inputminted[firstline=1,firstnumber=1,fontsize=\footnotesize,frame=single,linenos,bgcolor=lightgray,label=job\_risk.ini]{ini}{oqum/risk/verbatim/config_classical_risk.ini}
  \caption{Example risk configuration file for a classical probabilistic risk calculation (\href{https://raw.githubusercontent.com/GEMScienceTools/oq-engine-docs/master/oqum/risk/verbatim/config_classical_risk.ini}{Download example})}
  \label{lst:config_classical_risk}
\end{listing}

Now, the above calculations described by the two configuration files
``job\_hazard.ini'' and ``job\_risk.ini'' can be run sequentially or
separately, as illustrated in Example~2 in
Section~\ref{sec:config_scenario_damage}. The new parameters introduced in the
above risk configuration file example
(Listing~\ref{lst:config_classical_risk}) are described below:

\begin{itemize}

	\item \Verb+lrem_steps_per_interval+: this parameter controls the number of
	  intermediate values between consecutive loss ratios (as defined in the 
	  \gls{vulnerabilitymodel}) that are considered in the risk calculations.
	  A larger number of loss ratios than those defined in each
	  \gls{vulnerabilityfunction} should be considered, in order to better
	  account for the uncertainty in the loss ratio distribution. If this
	  parameter is not defined in the configuration file, the \glsdesc{acr:oqe}
	  assumes the \Verb+lrem_steps_per_interval+ to be equal to 5. More details
	  are provided in the OpenQuake Book (Risk).

	\item \Verb+quantile_loss_curves+: this parameter can be used to request
	  the computation of quantile loss curves for computations involving
	  non-trivial logic trees. The quantiles for which the loss curves should
	  be computed must be provided as a comma separated list. If this parameter
	  is not included in the configuration file, quantile loss curves will not
	  be computed. 

	\item \Verb+conditional_loss_poes+: this parameter can be used to request
	  the computation of probabilistic loss maps, which give the loss levels
	  exceeded at the specified probabilities of exceedance over the time
	  period specified by \Verb+risk_investigation_time+. The probabilities of
	  exceedance for which the loss maps should be computed must be provided as
	  a comma separated list. If this parameter is not included in the
	  configuration file, probabilistic loss maps will not be computed.

\end{itemize}

\section{Stochastic Event Based Seismic Risk Calculator}
\label{sec:config_event_based_risk}
The parameter \Verb+calculation_mode+ needs to be set to
\Verb+event_based_risk+ in order to use this calculator.

Most of the job configuration parameters required for running a stochastic
event based risk calculation are the same as those described in the previous
sections for the scenario risk calculator and the classical probabilistic risk
calculator. The remaining parameters specific to the stochastic event based
risk calculator are illustrated through the example below.


\paragraph{Example 1}

This example illustrates a stochastic event based risk calculation which uses
a single configuration file to first compute the \glspl{acr:ses} and
\glspl{acr:gmf} for the given source model and ground motion model, and then
calculate event loss tables, loss exceedance curves and probabilistic loss
maps for structural losses, nonstructural losses, insured structural losses,
and occupants, based on the \glspl{acr:gmf}. The job configuration file
required for running this stochastic event based risk calculation is shown in
Listing~\ref{lst:config_event_based_risk_combined}.

\begin{listing}[htbp]
  \inputminted[firstline=1,firstnumber=1,fontsize=\scriptsize
  ,frame=single,bgcolor=lightgray,linenos,label=job.ini]{ini}{oqum/risk/verbatim/config_event_based_risk_combined.ini}
  \caption{Example combined configuration file for running a stochastic event based risk calculation (\href{https://raw.githubusercontent.com/GEMScienceTools/oq-engine-docs/master/oqum/risk/verbatim/config_event_based_risk_combined.ini}{Download example})}
  \label{lst:config_event_based_risk_combined}
\end{listing}

Similar to that the procedure described for the Scenario Risk calculator, a
Monte Carlo sampling process is also employed in this calculator to take into
account the uncertainty in the conditional loss ratio at a particular
intensity level. Hence, the parameters \Verb+asset_correlation+ and
\Verb+master_seed+ may be defined as previously described for the Scenario
Risk calculator in Section~\ref{sec:config_scenario_risk}. This calculator is
also capable of estimating insured losses and therefore, setting the
\Verb+insured_losses+ attribute to \Verb+true+ will generate all results (loss
tables, loss curves, loss maps) for insured losses as well. The parameter
``risk\_investigation\_time'' specifies the time period for which the event
loss tables and loss exceedance curves will be calculated, similar to the
Classical Probabilistic Risk calculator. If this parameter is not provided in
the risk job configuration file, the time period used is the same as that
specifed in the hazard calculation using the parameter ``investigation\_time''.

The new parameters introduced in this example are described below:

\begin{itemize}

  \item \Verb+minimum_intensity+: this optional parameter specifies the minimum
    intensity levels for each of the intensity measure types in the risk model.
    Ground motion fields where each ground motion value is less than the 
    specified minimum threshold are discarded. This helps speed up calculations
    and reduce memory consumption by considering only those ground motion fields
    that are likely to contribute to losses. It is also possible to set the same
    threshold value for all intensity measure types by simply providing a single
    value to this parameter. For instance: ``minimum\_intensity = 0.05'' would
    set the threshold to 0.05 g for all intensity measure types in the risk 
    calculation.
    If this parameter is not set, the \glsdesc{acr:oqe} extracts the minimum
    thresholds for each intensity measure type from the vulnerability
    models provided, picking the first intensity value for which the mean loss
    ratio is nonzero.

  \item \Verb+loss_curve_resolution+: this parameter specifies the number of
    points on the aggregate loss curve. The loss levels on the aggregate loss
    curve are obtained by dividing the interval between the minimum and maximum
    portfolio losses in the portfolio loss table into `n' equispaced intervals,
    where `n' is the value specified for the \Verb+loss_curve_resolution+.
    If this parameter is not set, the \glsdesc{acr:oqe} uses a default value of
    20 for the \Verb+loss_curve_resolution+.

  \item \Verb+loss_ratios+: this parameter specifies the set of loss ratios at
    which the individual asset loss curves will be computed. If
    \Verb+loss_ratios+ is not set in the configuration file, the individual 
    asset loss curves will not be computed; and only the aggregate loss curve
    for the portfolio of assets will be computed. Furthermore, if
    \Verb+loss_ratios+ is not set in the configuration file, loss maps will
    also \textbf{not} be computed.

  \item \Verb+avg_losses+: this boolean parameter specifies whether the average
    asset losses over the time period ``risk\_investigation\_time'' should be
    computed. The default value of this parameter is \Verb+false+.

    \begin{equation*}
    \begin{split}
    average\_loss & = sum(event\_losses) \\
                 & \div (hazard\_investigation\_time \times ses\_per\_logic\_tree\_path) \\
                 & \times risk\_investigation\_time
    \end{split}
    \end{equation*}

\end{itemize}

The above calculation can be run using the command line:

\begin{minted}[fontsize=\footnotesize,frame=single,bgcolor=lightgray]{shell-session}
user@ubuntu:~\$ oq engine --run job.ini
\end{minted}

Computation of the loss tables, loss curves, and average losses for each
individual \gls{asset} in the \gls{exposuremodel} can be resource intensive,
and thus these outputs are not generated by default, unless instructed to by
using the parameters described above.


\section{Retrofit Benefit-Cost Ratio Calculator}
\label{sec:config_benefit_cost}
As previously explained, this calculator uses loss exceedance curves which are
calculated using the Classical Probabilistic risk calculator. In order to run
this calculator, the parameter \Verb+calculation_mode+ needs to be set to
\Verb+classical_bcr+.

Most of the job configuration parameters required for running a classical
retrofit benefit-cost ratio calculation are the same as those described in the
previous section for the classical probabilistic risk calculator. The
remaining parameters specific to the classical retrofit benefit-cost ratio
calculator are illustrated through the examples below.

\paragraph{Example 1}

This example illustrates a classical probabilistic retrofit benefit-cost ratio
calculation which uses a single configuration file to first compute the hazard
curves for the given source model and ground motion model, then calculate loss
exceedance curves based on the hazard curves using both the original
vulnerability model and the vulnerability model for the retrofitted
structures, then calculate the reduction in average annual losses due to the
retrofits, and finally calculate the benefit-cost ratio for each asset. A
minimal job configuration file required for running a classical probabilistic
retrofit benefit-cost ratio calculation is shown in
Listing~\ref{lst:config_classical_bcr_combined}.

\begin{listing}[htbp]
  \inputminted[firstline=1,firstnumber=1,fontsize=\footnotesize,frame=single,linenos,bgcolor=lightgray,label=job.ini]{ini}{oqum/risk/verbatim/config_classical_bcr_combined.ini}
  \caption{Example configuration file for a classical probabilistic retrofit benefit-cost ratio calculation (\href{https://raw.githubusercontent.com/GEMScienceTools/oq-engine-docs/master/oqum/risk/verbatim/config_classical_bcr_combined.ini}{Download example})}
  \label{lst:config_classical_bcr_combined}
\end{listing}

The new parameters introduced in the above example configuration file are
described below:

\begin{itemize}

  \item \Verb+vulnerability_retrofitted_file+: this parameter is used to
    specify the path to the \gls{vulnerabilitymodel} file containing the
    \glspl{vulnerabilityfunction} for the retrofitted asset

  \item \Verb+interest_rate+: this parameter is used in the calculation of the
    present value of potential future benefits by discounting future cash flows

  \item \Verb+asset_life_expectancy+: this variable defines the life
    expectancy or design life of the assets, and is used as the time-frame in
    which the costs and benefits of the retrofit will be compared

\end{itemize}

The above calculation can be run using the command line:

\begin{minted}[fontsize=\footnotesize,frame=single,bgcolor=lightgray]{shell-session}
user@ubuntu:~\$ oq engine --run job.ini
\end{minted}

After the calculation is completed, a message similar to the following will be
displayed:

\begin{minted}[fontsize=\footnotesize,frame=single,bgcolor=lightgray]{shell-session}
Calculation 2776 completed in 25 seconds. Results:
  id | name
5422 | Benefit-cost ratio distribution | BCR Map. type=structural, hazard=5420
\end{minted}

\section{Exporting Risk Results}
\label{sec:risk_export}
To obtain a list of all risk calculations that have been previously run
(successfully or unsuccessfully), or that are currently running, the following
command can be employed:

\begin{minted}[fontsize=\footnotesize,frame=single,bgcolor=lightgray]{shell-session}
user@ubuntu:~\$ oq engine --list-risk-calculations
\end{minted}

or simply:

\begin{minted}[fontsize=\footnotesize,frame=single,bgcolor=lightgray]{shell-session}
user@ubuntu:~\$ oq engine --lrc
\end{minted}

Which will display a list of risk calculations as presented below.

\begin{minted}[fontsize=\footnotesize,frame=single,bgcolor=lightgray]{shell-session}
job_id |     status |          start_time |     description
     1 |   complete | 2015-12-02 08:50:30 | Scenario damage example
     2 |     failed | 2015-12-03 09:56:17 | Scenario risk example
     3 |   complete | 2015-12-04 10:45:32 | Scenario risk example
     4 |   complete | 2015-12-04 10:48:33 | Classical risk example
     5 |   complete | 2020-07-09 13:47:45 | Event based risk aggregation example     
\end{minted}

Then, in order to display a list of the risk outputs from a given job which has
completed successfully, the following command can be used:

\begin{minted}[fontsize=\footnotesize,frame=single,bgcolor=lightgray]{shell-session}
user@ubuntu:~\$ oq engine --list-outputs <risk_calculation_id>
\end{minted}

or simply:

\begin{minted}[fontsize=\footnotesize,frame=single,bgcolor=lightgray]{shell-session}
user@ubuntu:~\$ oq engine --lo <risk_calculation_id>
\end{minted}

which will display a list of outputs for the calculation requested, as
presented below:

\begin{minted}[fontsize=\footnotesize,frame=single,bgcolor=lightgray]{shell-session}
Calculation 5 results:
  id | name
  11 | Aggregate Event Losses
   1 | Aggregate Loss Curves
   2 | Aggregate Loss Curves Statistics
   3 | Aggregate Losses
   4 | Aggregate Losses Statistics
   5 | Average Asset Losses Statistics
  13 | Earthquake Ruptures
   6 | Events
   7 | Full Report
  10 | Input Files
  12 | Realizations
  14 | Source Loss Table
  15 | Total Loss Curves
  16 | Total Loss Curves Statistics
  17 | Total Losses
  18 | Total Losses Statistics
\end{minted}

Then, in order to export all of the risk calculation outputs in the
default file format (csv for most outputs), the following command can be used:

\begin{minted}[fontsize=\footnotesize,frame=single,bgcolor=lightgray]{shell-session}
user@ubuntu:~\$ oq engine --export-outputs <risk_calculation_id> <output_directory>
\end{minted}

or simply:

\begin{minted}[fontsize=\footnotesize,frame=single,bgcolor=lightgray]{shell-session}
user@ubuntu:~\$ oq engine --eos <risk_calculation_id> <output_directory>
\end{minted}

If, instead of exporting all of the outputs from a particular calculation,
only particular output files need to be exported, this can be achieved by
using the \Verb+--export-output+ option and providing the id of the required
output:

\begin{minted}[fontsize=\footnotesize,frame=single,bgcolor=lightgray]{shell-session}
user@ubuntu:~\$ oq engine --export-output <risk_output_id> <output_directory>
\end{minted}

or simply:

\begin{minted}[fontsize=\footnotesize,frame=single,bgcolor=lightgray]{shell-session}
user@ubuntu:~\$ oq engine --eo <risk_output_id> <output_directory>
\end{minted}


\cleardoublepage