The same hazard input as described in the Classical Probabilistic Damage demo
is used for this demo. Thus, the workflow to produce the set of hazard curves
described in Section~\ref{sec:demos_classical_damage} is also valid herein.
Then, to run the Classical Probabilistic Risk demo, users should navigate to
the folder containing the demo input models and configuration files and employ
the following command:

\begin{minted}[fontsize=\footnotesize,frame=single,bgcolor=lightgray]{shell-session}
user@ubuntu:~\$ oq engine --run job_hazard.ini
\end{minted}

which will produce the following hazard output:

\begin{minted}[fontsize=\footnotesize,frame=single,bgcolor=lightgray]{shell-session}
Calculation 8971 completed in 34 seconds. Results:
  id | name
9074 | Hazard Curves
9075 | Realizations
\end{minted}

In this demo, loss exceedance curves for each asset and two probabilistic loss
maps (for probabilities of exceedance of 1\% and 10\%) are produced. The
following command launches these risk calculations:

\begin{minted}[fontsize=\footnotesize,frame=single,bgcolor=lightgray]{shell-session}
user@ubuntu:~\$ oq engine --run job_risk.ini --hc 8971
\end{minted}

and the following outputs are expected:

\begin{minted}[fontsize=\footnotesize,frame=single,bgcolor=lightgray]{shell-session}
Calculation 8973 completed in 16 seconds. Results:
  id | name
9077 | Asset Loss Curves Statistics
9078 | Asset Loss Maps Statistics
9079 | Average Asset Loss Statistics
\end{minted}
