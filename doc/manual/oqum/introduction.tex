OpenQuake-engine is the seismic hazard and risk calculation software developed by
the \glsdesc{acr:gem}. By following current standards in software
developments like test-driven development and continuous integration, the
\glsdesc{acr:oqe} aims at becoming an open, and community-driven tool for
seismic hazard and risk analysis.

The source code of the \glsdesc{acr:oqe} is available on a public web-based
repository at the following address:
\href{http://github.com/gem/oq-engine}{http://github.com/gem/oq-engine}.

The \glsdesc{acr:oqe} is available for the Linux, macOS, and Windows
platforms. It can be installed in several different ways. The following page
provides a handy guide for users to choose the most appropriate installation
method depending on their intended use cases:

\href{https://github.com/gem/oq-engine/blob/master/doc/installing/overview.md}{https://github.com/gem/oq-engine/blob/master/doc/installing/overview.md}.

This manual is for the command line interface for the \glsdesc{acr:oqe}. 

Guidance instructions for using the \glsdesc{acr:oqe} WebUI are available 
at \href{https://github.com/gem/oq-engine/blob/master/doc/running/server.md}{https://github.com/gem/oq-engine/blob/master/doc/running/server.md}.

A user manual for the QGIS plugin for the \glsdesc{acr:oqe} is available at 
\href{https://docs.openquake.org/oq-irmt-qgis/latest/}{https://docs.openquake.org/oq-irmt-qgis/latest/}. 
In particular, instructions for using the plugin as an interface for running \glsdesc{acr:oqe}
calculations are listed in Chapter 14, and methods for using the plugin for visualization 
of hazard and risk outputs are listed in Chapter 15.

An \gls{acr:oqe} analysis is launched from the command line of a terminal.

A schematic list of the options that can be used for the execution of the
\gls{acr:oqe} can be obtained with the following command:

\begin{minted}[fontsize=\footnotesize,frame=single,bgcolor=lightgray]{shell-session}
user@ubuntu:~\$ oq engine --help
\end{minted}

The result is the following:
\inputminted[firstline=1,fontsize=\footnotesize,frame=single]{shell-session}{oqum/help.txt}
